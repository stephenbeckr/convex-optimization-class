\documentclass[class=article,crop=false]{standalone} 
%Fall 2020
% Some basic packages
\usepackage{standalone}[subpreambles=true]
\usepackage[utf8]{inputenc}
\usepackage[T1]{fontenc}
\usepackage{textcomp}
\usepackage[english]{babel}
\usepackage{url}
\usepackage{graphicx}
\usepackage{float}
\usepackage{enumitem}
\usepackage{lmodern}
\usepackage{hyperref}
\usepackage[usenames,svgnames,dvipsnames]{xcolor}


\pdfminorversion=7

% Don't indent paragraphs, leave some space between them
\usepackage{parskip}

% Hide page number when page is empty
\usepackage{emptypage}
\usepackage{subcaption}
\usepackage{multicol}
\usepackage[dvipsnames]{xcolor}
\usepackage[b]{esvect}

% Math stuff
\usepackage{amsmath, amsfonts, mathtools, amsthm, amssymb}
\usepackage{bbm}

% Fancy script capitals
\usepackage{mathrsfs}
\usepackage{cancel}
% Bold math
\usepackage{bm}
% Some shortcuts
\newcommand{\rr}{\ensuremath{\mathbb{R}}}
\newcommand{\zz}{\ensuremath{\mathbb{Z}}}
\newcommand{\qq}{\ensuremath{\mathbb{Q}}}
\newcommand{\nn}{\ensuremath{\mathbb{N}}}
\newcommand{\ff}{\ensuremath{\mathbb{F}}}
\newcommand{\cc}{\ensuremath{\mathbb{C}}}
\newcommand{\ee}{\ensuremath{\mathbb{E}}}
\renewcommand\O{\ensuremath{\emptyset}}
\newcommand{\norm}[1]{{\left\lVert{#1}\right\rVert}}
\newcommand{\ve}[1]{{\mathbf{#1}}}
\newcommand\allbold[1]{{\boldmath\textbf{#1}}}
\DeclareMathOperator{\lcm}{lcm}
\DeclareMathOperator{\im}{im}
\DeclareMathOperator{\coim}{coim}
\DeclareMathOperator{\dom}{dom}
\DeclareMathOperator{\tr}{tr}
\DeclareMathOperator{\rank}{rank}
\DeclareMathOperator*{\var}{Var}
\DeclareMathOperator*{\ev}{E}
\DeclareMathOperator{\sinc}{sinc}
\DeclareMathOperator{\dg}{deg}
\DeclareMathOperator{\aff}{aff}
\DeclareMathOperator{\conv}{conv}
\DeclareMathOperator{\epi}{epi}
\DeclareMathOperator{\inte}{int}
\DeclareMathOperator{\ri}{ri}
\DeclareMathOperator*{\argmin}{argmin}
\DeclareMathOperator*{\argmax}{argmax}
\DeclareMathOperator{\graph}{graph}
\DeclareMathOperator{\sgn}{sgn}
\DeclareMathOperator*{\Rep}{Rep}
\DeclareMathOperator{\Proj}{Proj}
\DeclareMathOperator{\prox}{prox}
\DeclareMathOperator{\mat}{mat}
\let\vec\relax
\DeclareMathOperator{\vec}{vec}
\let\Re\relax
\DeclareMathOperator{\Re}{Re}
\let\Im\relax
\DeclareMathOperator{\Im}{Im}
% Put x \to \infty below \lim
\let\svlim\lim\def\lim{\svlim\limits}

%wide hat
\usepackage{scalerel,stackengine}
\stackMath
\newcommand*\wh[1]{%
\savestack{\tmpbox}{\stretchto{%
  \scaleto{%
    \scalerel*[\widthof{\ensuremath{#1}}]{\kern-.6pt\bigwedge\kern-.6pt}%
    {\rule[-\textheight/2]{1ex}{\textheight}}%WIDTH-LIMITED BIG WEDGE
  }{\textheight}% 
}{0.5ex}}%
\stackon[1pt]{#1}{\tmpbox}%
}
\parskip 1ex

%Make implies and impliedby shorter
\let\implies\Rightarrow
\let\impliedby\Leftarrow
\let\iff\Leftrightarrow
\let\epsilon\varepsilon

% Add \contra symbol to denote contradiction
\usepackage{stmaryrd} % for \lightning
\newcommand\contra{\scalebox{1.5}{$\lightning$}}

% \let\phi\varphi

% Command for short corrections
% Usage: 1+1=\correct{3}{2}

\definecolor{correct}{HTML}{009900}
\newcommand\correct[2]{\ensuremath{\:}{\color{red}{#1}}\ensuremath{\to }{\color{correct}{#2}}\ensuremath{\:}}
\newcommand\green[1]{{\color{correct}{#1}}}

% horizontal rule
\newcommand\hr{
    \noindent\rule[0.5ex]{\linewidth}{0.5pt}
}

% hide parts
\newcommand\hide[1]{}

% si unitx
\usepackage{siunitx}
\sisetup{locale = FR}

%allows pmatrix to stretch
\makeatletter
\renewcommand*\env@matrix[1][\arraystretch]{%
  \edef\arraystretch{#1}%
  \hskip -\arraycolsep
  \let\@ifnextchar\new@ifnextchar
  \array{*\c@MaxMatrixCols c}}
\makeatother

\renewcommand{\arraystretch}{0.8}

% Environments
\makeatother
% For box around Definition, Theorem, \ldots
%%fakesection Theorems
\usepackage{thmtools}
\usepackage[framemethod=TikZ]{mdframed}

\theoremstyle{definition}
\mdfdefinestyle{mdbluebox}{%
	roundcorner = 10pt,
	linewidth=1pt,
	skipabove=12pt,
	innerbottommargin=9pt,
	skipbelow=2pt,
	nobreak=true,
	linecolor=blue,
	backgroundcolor=TealBlue!5,
}
\declaretheoremstyle[
	headfont=\sffamily\bfseries\color{MidnightBlue},
	mdframed={style=mdbluebox},
	headpunct={\\[3pt]},
	postheadspace={0pt}
]{thmbluebox}

\mdfdefinestyle{mdredbox}{%
	linewidth=0.5pt,
	skipabove=12pt,
	frametitleaboveskip=5pt,
	frametitlebelowskip=0pt,
	skipbelow=2pt,
	frametitlefont=\bfseries,
	innertopmargin=4pt,
	innerbottommargin=8pt,
	nobreak=false,
	linecolor=RawSienna,
	backgroundcolor=Salmon!5,
}
\declaretheoremstyle[
	headfont=\bfseries\color{RawSienna},
	mdframed={style=mdredbox},
	headpunct={\\[3pt]},
	postheadspace={0pt},
]{thmredbox}

\declaretheorem[%
style=thmbluebox,name=Theorem,numberwithin=section]{thm}
\declaretheorem[style=thmbluebox,name=Lemma,sibling=thm]{lem}
\declaretheorem[style=thmbluebox,name=Proposition,sibling=thm]{prop}
\declaretheorem[style=thmbluebox,name=Corollary,sibling=thm]{coro}
\declaretheorem[style=thmredbox,name=Example,sibling=thm]{eg}

\mdfdefinestyle{mdgreenbox}{%
	roundcorner = 10pt,
	linewidth=1pt,
	skipabove=12pt,
	innerbottommargin=9pt,
	skipbelow=2pt,
	nobreak=true,
	linecolor=ForestGreen,
	backgroundcolor=ForestGreen!5,
}

\declaretheoremstyle[
	headfont=\bfseries\sffamily\color{ForestGreen!70!black},
	bodyfont=\normalfont,
	spaceabove=2pt,
	spacebelow=1pt,
	mdframed={style=mdgreenbox},
	headpunct={ --- },
]{thmgreenbox}

\declaretheorem[style=thmgreenbox,name=Definition,sibling=thm]{defn}

\mdfdefinestyle{mdgreenboxsq}{%
	linewidth=1pt,
	skipabove=12pt,
	innerbottommargin=9pt,
	skipbelow=2pt,
	nobreak=true,
	linecolor=ForestGreen,
	backgroundcolor=ForestGreen!5,
}
\declaretheoremstyle[
	headfont=\bfseries\sffamily\color{ForestGreen!70!black},
	bodyfont=\normalfont,
	spaceabove=2pt,
	spacebelow=1pt,
	mdframed={style=mdgreenboxsq},
	headpunct={},
]{thmgreenboxsq}
\declaretheoremstyle[
	headfont=\bfseries\sffamily\color{ForestGreen!70!black},
	bodyfont=\normalfont,
	spaceabove=2pt,
	spacebelow=1pt,
	mdframed={style=mdgreenboxsq},
	headpunct={},
]{thmgreenboxsq*}

\mdfdefinestyle{mdblackbox}{%
	skipabove=8pt,
	linewidth=3pt,
	rightline=false,
	leftline=true,
	topline=false,
	bottomline=false,
	linecolor=black,
	backgroundcolor=RedViolet!5!gray!5,
}
\declaretheoremstyle[
	headfont=\bfseries,
	bodyfont=\normalfont\small,
	spaceabove=0pt,
	spacebelow=0pt,
	mdframed={style=mdblackbox}
]{thmblackbox}

\theoremstyle{plain}
\declaretheorem[name=Question,sibling=thm,style=thmblackbox]{ques}
\declaretheorem[name=Remark,sibling=thm,style=thmgreenboxsq]{remark}
\declaretheorem[name=Remark,sibling=thm,style=thmgreenboxsq*]{remark*}

\theoremstyle{definition}
\newtheorem{claim}[thm]{Claim}
\theoremstyle{remark}
\newtheorem*{case}{Case}
\newtheorem*{notation}{Notation}
\newtheorem*{note}{Note}
\newtheorem*{motivation}{Motivation}
\newtheorem*{intuition}{Intuition}

% Make section starts with 1 for report type
%\renewcommand\thesection{\arabic{section}}

% End example and intermezzo environments with a small diamond (just like proof
% environments end with a small square)
\usepackage{etoolbox}
\AtEndEnvironment{vb}{\null\hfill$\diamond$}%
\AtEndEnvironment{intermezzo}{\null\hfill$\diamond$}%
% \AtEndEnvironment{opmerking}{\null\hfill$\diamond$}%

% Fix some spacing
% http://tex.stackexchange.com/questions/22119/how-can-i-change-the-spacing-before-theorems-with-amsthm
\makeatletter
\def\thm@space@setup{%
  \thm@preskip=\parskip \thm@postskip=0pt
}

% Fix some stuff
% %http://tex.stackexchange.com/questions/76273/multiple-pdfs-with-page-group-included-in-a-single-page-warning
\pdfsuppresswarningpagegroup=1

\renewcommand{\baselinestretch}{1.5}
\RequirePackage{hyperref}[6.83]
\hypersetup{
  colorlinks=false,
  frenchlinks=false,
  pdfborder={0 0 0},
  naturalnames=false,
  hypertexnames=false,
  breaklinks
}
\urlstyle{same}

\usepackage{graphics}
\usepackage{epstopdf}

%%
%% Add support for color in order to color the hyperlinks.
%% 
\hypersetup{
  colorlinks = true,
  allcolors = siaminlinkcolor,
  urlcolor = siamexlinkcolor,
}
%%fakesection Links
\hypersetup{
    colorlinks,
    linkcolor={red!50!black},
    citecolor={green!50!black},
    urlcolor={blue!80!black}
}
%customization of cleveref
\RequirePackage[capitalize,nameinlink]{cleveref}[0.19]

% Per SIAM Style Manual, "section" should be lowercase
\crefname{section}{section}{sections}
\crefname{subsection}{subsection}{subsections}
\Crefname{section}{Section}{Sections}
\Crefname{subsection}{Subsection}{Subsections}

% Per SIAM Style Manual, "Figure" should be spelled out in references
\Crefname{figure}{Figure}{Figures}

% Per SIAM Style Manual, don't say equation in front on an equation.
\crefformat{equation}{\textup{#2(#1)#3}}
\crefrangeformat{equation}{\textup{#3(#1)#4--#5(#2)#6}}
\crefmultiformat{equation}{\textup{#2(#1)#3}}{ and \textup{#2(#1)#3}}
{, \textup{#2(#1)#3}}{, and \textup{#2(#1)#3}}
\crefrangemultiformat{equation}{\textup{#3(#1)#4--#5(#2)#6}}%
{ and \textup{#3(#1)#4--#5(#2)#6}}{, \textup{#3(#1)#4--#5(#2)#6}}{, and \textup{#3(#1)#4--#5(#2)#6}}

% But spell it out at the beginning of a sentence.
\Crefformat{equation}{#2Equation~\textup{(#1)}#3}
\Crefrangeformat{equation}{Equations~\textup{#3(#1)#4--#5(#2)#6}}
\Crefmultiformat{equation}{Equations~\textup{#2(#1)#3}}{ and \textup{#2(#1)#3}}
{, \textup{#2(#1)#3}}{, and \textup{#2(#1)#3}}
\Crefrangemultiformat{equation}{Equations~\textup{#3(#1)#4--#5(#2)#6}}%
{ and \textup{#3(#1)#4--#5(#2)#6}}{, \textup{#3(#1)#4--#5(#2)#6}}{, and \textup{#3(#1)#4--#5(#2)#6}}

% Make number non-italic in any environment.
\crefdefaultlabelformat{#2\textup{#1}#3}

% My name
\author{Jaden Wang}



\begin{document}
\subsection{by hand}
\subsubsection{implicit differentiation}

$ F(x,y) = 0$ often implicitly defines a function  $ y = f(x)$. Find  $ dy /dx$.

 $ F = 0 \implies dF / dx =  d0 / dx =0$. Thus,
 \begin{align*}
 	0=\frac{d F}{d x} &= \frac{\partial F}{\partial x} \frac{d x}{d x} + \frac{\partial F}{\partial y} \frac{d y}{d x}  \\
	0&= \frac{\partial F}{\partial x} + \frac{\partial F}{\partial y} \frac{dy}{dx} \\
	\frac{d y}{d x} &= - \frac{ \frac{\partial F}{\partial x} }{ \frac{\partial F}{\partial y} }
 \end{align*}

\subsubsection{matrix variables}
See BV04, or A Matrix Handbook for Statisticians Seber 08, or Matrix Cookbook, for examples.
\begin{eg}
\begin{align*}
	\nabla (\log \det (X)) = X^{-1}, X \succ 0
\end{align*}
\end{eg}
\subsubsection{parametric functions}
See notes for details.
\begin{align*}
	f(x) = \max_{z \in Z} \phi(x,z)
\end{align*}
\begin{thm}[Danskin]
	Suppose $ Z$ compact,  $ \phi$ jointly continuous and $ \phi( \cdot ,z)$ convex. Define
	\begin{align*}
		Z(x) = \argmax_{z \in Z} \phi(x,z).
	\end{align*}
	Then
	\begin{enumerate}[label=(\arabic*)]
		\item The directional derivative $ D_d$ satisfies
			 \begin{align*}
				 D_d f(x) = \max_{z \in Z(x)} D_d \phi(x,z)
			\end{align*}
		\item If $ \phi( \cdot ,z)$ is differentiable, $ \nabla _x \phi$ is continuous, then
			\begin{align*}
				\partial f(x) = \conv \{\nabla _x \phi(x,z): z \in Z(x)\} 
			\end{align*}
			and $ Z(x)$ a singleton  $ \implies$ $ f$ is differentiable.
	\end{enumerate}
\end{thm}
\begin{note}
This theorem doesn't apply to the discrete case.
\end{note}
\begin{thm}[Dubovitskii and Milyutin]
	If $ |Z|$ is finite,  $ \phi( \cdot ,z)$ is convex $ \ \forall \ z \in Z$, then
	\begin{align*}
		\partial f = \conv \left\{ \bigcup_{ z \in Z(x)} \partial \phi(x,z) \right\} 
	\end{align*}
\end{thm}
\begin{eg}
	$ f(x) = |x| = \max \{x,-x\} $. Then $ \partial f(0) = \conv \{1,-1\} = [-1,1] $.
\end{eg}

\begin{align*}
	f(x) = \min_{z \in Z} \phi(x,z)
\end{align*}
\begin{thm}
Under the same conditions,
\begin{align*}
	\partial f(x) = \partial \phi(x,z) \ \forall \ z \in Z(x)
\end{align*}
\end{thm}

\begin{align*}
	f(x) = \int \phi(x,z) dz
\end{align*}
\begin{thm}
Sometimes,
\begin{align*}
	f'(x) = \int \frac{d}{dx} \phi(x,z) dz 
\end{align*}
\end{thm}
\subsection{Adjoint state method}
\begin{enumerate}[label=(\arabic*)]
	\item implicit differentiation and careful parentheses.
		\begin{align*}
			\min_{p} g(u(p),p)\ s.t.\ f(u(p),p) = 0, u \in \mathcal{ H}
		\end{align*}
		\begin{eg}
			$ u_{tt} = c^2(x) u_{x x}$. Let $p= c^2(x)$, \emph{e.g.} varying speed of sound. Or $ p$ could be other parameters like ICs and BCs. Applications in inverses problems such as oil detection. This is often called ``PDE-constrained optimization problem''.
			To solve this, we ignore the constraint first. Then we wish to solve $ dg /dp$.
		\end{eg}
		\begin{eg}
		\begin{align*}
			\min g(u,p) &\ s.t.\ A u = b, A= V \text{ diag}  (p) V^{T}, p \in \rr^{n}, u \in \rr^{m}, A \in \rr^{m} \times \rr^{m}\\
			g(u,p) &= \frac{1}{2} \sum_i (u_i - y_i)^2 + \frac{1}{2} \norm{ p}^2 \\
		\end{align*}
		\end{eg}
		\begin{eg}
			Let $ \mathcal{ H} = \rr^{m}$. $ f(u,p) = A(p) \cdot  u -b(p)=0$, linear in $ u$, so $ u(p) = A(p)^{-1} b(p)$. The goal is to find
			\begin{align*}
				(\nabla _p g)^{T} &= g_p + g_u u_p \\
				0=f_p &= A_p u + Au_p -b_p \\
				u_p &= A^{-1}(b_p - A_p u)\\
				(\nabla _p g)^{T} &= g_p + g_u (A^{-1}(b_p - A_p u))\\
						  &= g_p + (g_u A^{-1})(b_p - A_p u) \\
			\end{align*}
			Let $ \lambda^* = g_u A^{-1} \implies \lambda = A^{-*} g_u^* $, then we have the adjoint-state equation
			\begin{align*}
				A^* \lambda = g_u^* 
			\end{align*}
			Now by clever grouping we are only solving one RHS instead of $ n$ RHS.
		\end{eg}
	\item adjoints of (bounded/unbounded) linear operators.
		What about adjoints if $ | \mathcal{ H}| = \infty$?
		\begin{eg}
			Let $ L : \mathcal{ H} \to \mathcal{ H}$, $ \mathcal{ H} = L^2[0,T]$:
		\begin{align*}
			L(u) = 3u'+4u,u(0)=0, t \in [0,T].
		\end{align*}
		Let $ f(u,p) = L(u) -h(t)$. To get the formal adjoint (adjoint doesnt exists since $ L$ doesn't have full domain),
		 \begin{align*}
			 \int_0^T (3u'+4u) v dt &= 3\int u'vdt + 4\int uv dt\\
			 \int u(-3v'+ 4v) dt&= 3uv\big|_0^T - 3\int uv' dt + 4 \int uv dt && \text{ set }  v(T)=0
		\end{align*}
		Thus,
		\begin{align*}
			L^* (v) = -3v'+4v, v(T)=0, L^* (v) = h
		\end{align*}
		\end{eg}
		\begin{remark}
		~\begin{enumerate}[label=(\arabic*)]
			\item doesn't require linear PDE's
			\item not always a good idea: memory issues, consistency from the order of applying discretization or optimization
			\item software: Dolfin-Adjoint, FEnICS
		\end{enumerate}
		\end{remark}
\end{enumerate}
\end{document}
