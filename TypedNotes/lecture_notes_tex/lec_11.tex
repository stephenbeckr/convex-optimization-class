\documentclass[class=article,crop=false]{standalone} 
%Fall 2020
% Some basic packages
\usepackage{standalone}[subpreambles=true]
\usepackage[utf8]{inputenc}
\usepackage[T1]{fontenc}
\usepackage{textcomp}
\usepackage[english]{babel}
\usepackage{url}
\usepackage{graphicx}
\usepackage{float}
\usepackage{enumitem}
\usepackage{lmodern}
\usepackage{hyperref}
\usepackage[usenames,svgnames,dvipsnames]{xcolor}


\pdfminorversion=7

% Don't indent paragraphs, leave some space between them
\usepackage{parskip}

% Hide page number when page is empty
\usepackage{emptypage}
\usepackage{subcaption}
\usepackage{multicol}
\usepackage[dvipsnames]{xcolor}
\usepackage[b]{esvect}

% Math stuff
\usepackage{amsmath, amsfonts, mathtools, amsthm, amssymb}
\usepackage{bbm}

% Fancy script capitals
\usepackage{mathrsfs}
\usepackage{cancel}
% Bold math
\usepackage{bm}
% Some shortcuts
\newcommand{\rr}{\ensuremath{\mathbb{R}}}
\newcommand{\zz}{\ensuremath{\mathbb{Z}}}
\newcommand{\qq}{\ensuremath{\mathbb{Q}}}
\newcommand{\nn}{\ensuremath{\mathbb{N}}}
\newcommand{\ff}{\ensuremath{\mathbb{F}}}
\newcommand{\cc}{\ensuremath{\mathbb{C}}}
\newcommand{\ee}{\ensuremath{\mathbb{E}}}
\renewcommand\O{\ensuremath{\emptyset}}
\newcommand{\norm}[1]{{\left\lVert{#1}\right\rVert}}
\newcommand{\ve}[1]{{\mathbf{#1}}}
\newcommand\allbold[1]{{\boldmath\textbf{#1}}}
\DeclareMathOperator{\lcm}{lcm}
\DeclareMathOperator{\im}{im}
\DeclareMathOperator{\coim}{coim}
\DeclareMathOperator{\dom}{dom}
\DeclareMathOperator{\tr}{tr}
\DeclareMathOperator{\rank}{rank}
\DeclareMathOperator*{\var}{Var}
\DeclareMathOperator*{\ev}{E}
\DeclareMathOperator{\sinc}{sinc}
\DeclareMathOperator{\dg}{deg}
\DeclareMathOperator{\aff}{aff}
\DeclareMathOperator{\conv}{conv}
\DeclareMathOperator{\epi}{epi}
\DeclareMathOperator{\inte}{int}
\DeclareMathOperator{\ri}{ri}
\DeclareMathOperator*{\argmin}{argmin}
\DeclareMathOperator*{\argmax}{argmax}
\DeclareMathOperator{\graph}{graph}
\DeclareMathOperator{\sgn}{sgn}
\DeclareMathOperator*{\Rep}{Rep}
\DeclareMathOperator{\Proj}{Proj}
\DeclareMathOperator{\prox}{prox}
\DeclareMathOperator{\mat}{mat}
\let\vec\relax
\DeclareMathOperator{\vec}{vec}
\let\Re\relax
\DeclareMathOperator{\Re}{Re}
\let\Im\relax
\DeclareMathOperator{\Im}{Im}
% Put x \to \infty below \lim
\let\svlim\lim\def\lim{\svlim\limits}

%wide hat
\usepackage{scalerel,stackengine}
\stackMath
\newcommand*\wh[1]{%
\savestack{\tmpbox}{\stretchto{%
  \scaleto{%
    \scalerel*[\widthof{\ensuremath{#1}}]{\kern-.6pt\bigwedge\kern-.6pt}%
    {\rule[-\textheight/2]{1ex}{\textheight}}%WIDTH-LIMITED BIG WEDGE
  }{\textheight}% 
}{0.5ex}}%
\stackon[1pt]{#1}{\tmpbox}%
}
\parskip 1ex

%Make implies and impliedby shorter
\let\implies\Rightarrow
\let\impliedby\Leftarrow
\let\iff\Leftrightarrow
\let\epsilon\varepsilon

% Add \contra symbol to denote contradiction
\usepackage{stmaryrd} % for \lightning
\newcommand\contra{\scalebox{1.5}{$\lightning$}}

% \let\phi\varphi

% Command for short corrections
% Usage: 1+1=\correct{3}{2}

\definecolor{correct}{HTML}{009900}
\newcommand\correct[2]{\ensuremath{\:}{\color{red}{#1}}\ensuremath{\to }{\color{correct}{#2}}\ensuremath{\:}}
\newcommand\green[1]{{\color{correct}{#1}}}

% horizontal rule
\newcommand\hr{
    \noindent\rule[0.5ex]{\linewidth}{0.5pt}
}

% hide parts
\newcommand\hide[1]{}

% si unitx
\usepackage{siunitx}
\sisetup{locale = FR}

%allows pmatrix to stretch
\makeatletter
\renewcommand*\env@matrix[1][\arraystretch]{%
  \edef\arraystretch{#1}%
  \hskip -\arraycolsep
  \let\@ifnextchar\new@ifnextchar
  \array{*\c@MaxMatrixCols c}}
\makeatother

\renewcommand{\arraystretch}{0.8}

% Environments
\makeatother
% For box around Definition, Theorem, \ldots
%%fakesection Theorems
\usepackage{thmtools}
\usepackage[framemethod=TikZ]{mdframed}

\theoremstyle{definition}
\mdfdefinestyle{mdbluebox}{%
	roundcorner = 10pt,
	linewidth=1pt,
	skipabove=12pt,
	innerbottommargin=9pt,
	skipbelow=2pt,
	nobreak=true,
	linecolor=blue,
	backgroundcolor=TealBlue!5,
}
\declaretheoremstyle[
	headfont=\sffamily\bfseries\color{MidnightBlue},
	mdframed={style=mdbluebox},
	headpunct={\\[3pt]},
	postheadspace={0pt}
]{thmbluebox}

\mdfdefinestyle{mdredbox}{%
	linewidth=0.5pt,
	skipabove=12pt,
	frametitleaboveskip=5pt,
	frametitlebelowskip=0pt,
	skipbelow=2pt,
	frametitlefont=\bfseries,
	innertopmargin=4pt,
	innerbottommargin=8pt,
	nobreak=false,
	linecolor=RawSienna,
	backgroundcolor=Salmon!5,
}
\declaretheoremstyle[
	headfont=\bfseries\color{RawSienna},
	mdframed={style=mdredbox},
	headpunct={\\[3pt]},
	postheadspace={0pt},
]{thmredbox}

\declaretheorem[%
style=thmbluebox,name=Theorem,numberwithin=section]{thm}
\declaretheorem[style=thmbluebox,name=Lemma,sibling=thm]{lem}
\declaretheorem[style=thmbluebox,name=Proposition,sibling=thm]{prop}
\declaretheorem[style=thmbluebox,name=Corollary,sibling=thm]{coro}
\declaretheorem[style=thmredbox,name=Example,sibling=thm]{eg}

\mdfdefinestyle{mdgreenbox}{%
	roundcorner = 10pt,
	linewidth=1pt,
	skipabove=12pt,
	innerbottommargin=9pt,
	skipbelow=2pt,
	nobreak=true,
	linecolor=ForestGreen,
	backgroundcolor=ForestGreen!5,
}

\declaretheoremstyle[
	headfont=\bfseries\sffamily\color{ForestGreen!70!black},
	bodyfont=\normalfont,
	spaceabove=2pt,
	spacebelow=1pt,
	mdframed={style=mdgreenbox},
	headpunct={ --- },
]{thmgreenbox}

\declaretheorem[style=thmgreenbox,name=Definition,sibling=thm]{defn}

\mdfdefinestyle{mdgreenboxsq}{%
	linewidth=1pt,
	skipabove=12pt,
	innerbottommargin=9pt,
	skipbelow=2pt,
	nobreak=true,
	linecolor=ForestGreen,
	backgroundcolor=ForestGreen!5,
}
\declaretheoremstyle[
	headfont=\bfseries\sffamily\color{ForestGreen!70!black},
	bodyfont=\normalfont,
	spaceabove=2pt,
	spacebelow=1pt,
	mdframed={style=mdgreenboxsq},
	headpunct={},
]{thmgreenboxsq}
\declaretheoremstyle[
	headfont=\bfseries\sffamily\color{ForestGreen!70!black},
	bodyfont=\normalfont,
	spaceabove=2pt,
	spacebelow=1pt,
	mdframed={style=mdgreenboxsq},
	headpunct={},
]{thmgreenboxsq*}

\mdfdefinestyle{mdblackbox}{%
	skipabove=8pt,
	linewidth=3pt,
	rightline=false,
	leftline=true,
	topline=false,
	bottomline=false,
	linecolor=black,
	backgroundcolor=RedViolet!5!gray!5,
}
\declaretheoremstyle[
	headfont=\bfseries,
	bodyfont=\normalfont\small,
	spaceabove=0pt,
	spacebelow=0pt,
	mdframed={style=mdblackbox}
]{thmblackbox}

\theoremstyle{plain}
\declaretheorem[name=Question,sibling=thm,style=thmblackbox]{ques}
\declaretheorem[name=Remark,sibling=thm,style=thmgreenboxsq]{remark}
\declaretheorem[name=Remark,sibling=thm,style=thmgreenboxsq*]{remark*}

\theoremstyle{definition}
\newtheorem{claim}[thm]{Claim}
\theoremstyle{remark}
\newtheorem*{case}{Case}
\newtheorem*{notation}{Notation}
\newtheorem*{note}{Note}
\newtheorem*{motivation}{Motivation}
\newtheorem*{intuition}{Intuition}

% Make section starts with 1 for report type
%\renewcommand\thesection{\arabic{section}}

% End example and intermezzo environments with a small diamond (just like proof
% environments end with a small square)
\usepackage{etoolbox}
\AtEndEnvironment{vb}{\null\hfill$\diamond$}%
\AtEndEnvironment{intermezzo}{\null\hfill$\diamond$}%
% \AtEndEnvironment{opmerking}{\null\hfill$\diamond$}%

% Fix some spacing
% http://tex.stackexchange.com/questions/22119/how-can-i-change-the-spacing-before-theorems-with-amsthm
\makeatletter
\def\thm@space@setup{%
  \thm@preskip=\parskip \thm@postskip=0pt
}

% Fix some stuff
% %http://tex.stackexchange.com/questions/76273/multiple-pdfs-with-page-group-included-in-a-single-page-warning
\pdfsuppresswarningpagegroup=1

\renewcommand{\baselinestretch}{1.5}
\RequirePackage{hyperref}[6.83]
\hypersetup{
  colorlinks=false,
  frenchlinks=false,
  pdfborder={0 0 0},
  naturalnames=false,
  hypertexnames=false,
  breaklinks
}
\urlstyle{same}

\usepackage{graphics}
\usepackage{epstopdf}

%%
%% Add support for color in order to color the hyperlinks.
%% 
\hypersetup{
  colorlinks = true,
  allcolors = siaminlinkcolor,
  urlcolor = siamexlinkcolor,
}
%%fakesection Links
\hypersetup{
    colorlinks,
    linkcolor={red!50!black},
    citecolor={green!50!black},
    urlcolor={blue!80!black}
}
%customization of cleveref
\RequirePackage[capitalize,nameinlink]{cleveref}[0.19]

% Per SIAM Style Manual, "section" should be lowercase
\crefname{section}{section}{sections}
\crefname{subsection}{subsection}{subsections}
\Crefname{section}{Section}{Sections}
\Crefname{subsection}{Subsection}{Subsections}

% Per SIAM Style Manual, "Figure" should be spelled out in references
\Crefname{figure}{Figure}{Figures}

% Per SIAM Style Manual, don't say equation in front on an equation.
\crefformat{equation}{\textup{#2(#1)#3}}
\crefrangeformat{equation}{\textup{#3(#1)#4--#5(#2)#6}}
\crefmultiformat{equation}{\textup{#2(#1)#3}}{ and \textup{#2(#1)#3}}
{, \textup{#2(#1)#3}}{, and \textup{#2(#1)#3}}
\crefrangemultiformat{equation}{\textup{#3(#1)#4--#5(#2)#6}}%
{ and \textup{#3(#1)#4--#5(#2)#6}}{, \textup{#3(#1)#4--#5(#2)#6}}{, and \textup{#3(#1)#4--#5(#2)#6}}

% But spell it out at the beginning of a sentence.
\Crefformat{equation}{#2Equation~\textup{(#1)}#3}
\Crefrangeformat{equation}{Equations~\textup{#3(#1)#4--#5(#2)#6}}
\Crefmultiformat{equation}{Equations~\textup{#2(#1)#3}}{ and \textup{#2(#1)#3}}
{, \textup{#2(#1)#3}}{, and \textup{#2(#1)#3}}
\Crefrangemultiformat{equation}{Equations~\textup{#3(#1)#4--#5(#2)#6}}%
{ and \textup{#3(#1)#4--#5(#2)#6}}{, \textup{#3(#1)#4--#5(#2)#6}}{, and \textup{#3(#1)#4--#5(#2)#6}}

% Make number non-italic in any environment.
\crefdefaultlabelformat{#2\textup{#1}#3}

% My name
\author{Jaden Wang}



\begin{document}
\subsection{Gradient descent}

Problem: we want to solve $ \min_{x} f(x)$, $ f:\rr^{n} \to \rr$, $ f \in \Gamma_0 (\rr^{n})$ (proper, lsc, convex) and $ \nabla f$ is $L$-Lipschitz continuous (strongly smooth).

\subsubsection{Attempt 1}
\[
	x_{k+1} = \argmin_{x} \left[ f(x_k) + \underbrace{ \langle \nabla f(x_k), x- x_k \rangle}_{q_k (x) \text{ 1st order surrogate} } \right]
.\]
Linearization is a common trick to simplify problems. However, this fails because $ \min_{x} q_k(x) = -\infty$ for a linear function (unless it's already optimal). We can fix this by add a compact constraint. Then it's called \allbold{Frank-Wolfe} or \allbold{conditional gradient}. We omit this discussion as it's a bit niche.

\subsubsection{Attempt 2}
Consider the 2nd order Taylor series:
\[
	x_{k+1} = \argmin_{x} \underbrace{ f(x_k) + \langle \nabla f(x_k) , x-x_k \rangle + \frac{1}{2} \langle x-x_k, \nabla ^2 f(x_k) (x-x_k) \rangle}_{q_k(x) \text{ quadratic surrogate} }
.\] 
Since $ f$ is convex,  $ \nabla ^2 f(x) \succeq 0 \implies q_k(x)$ is a convex quadratic (sum of convex functions). 

To minimize $ q_k(x)$, we use Fermat's rule:
\begin{align*}
	0 &= \nabla q_k(x) \\
	  &= \nabla f(x_k) + \nabla ^2 f(x_k) (x-x_k) &&\text{ gradients of linear and quadratic terms} \\
	x_{k+1} &= x_k - \nabla ^2 f(x_k)^{-1} \nabla f(x_k) 
\end{align*}

This is \allbold{Newton's method}, a generalization of the "Newton-Raphson" for 1D root-finding, applied to the gradient. It is a \allbold{2nd-order method} because it involves the derivative of the gradient which is the second derivative (Hessian). 

\begin{remark}
	Unlike 1D root-finding, 2nd order methods in higher dimensions converge quickly but each iteration may be costly because we need to invert the Hessian and solving system of equations. This is about $ \mathcal{ O}(n^3)$. 1st-order methods only use $ \nabla f(x)$ and usually converge more slowly but each step is cheap at about $ \mathcal{ O}(n)$.
\end{remark}

\subsubsection{What to use?}
It depends:
\begin{itemize}
	\item Structure matters (is $ \nabla ^2 f$ easy to invert? Is it ill-conditioned (which hurts 1st order more)?)
	\item For small/medium problem size, high accuracy, we use 2nd order. This is default for cvx/cvxpy.
	\item In between problems: try both?
\end{itemize}

\subsubsection{Other types}
\begin{itemize}
	\item 3rd order: usually not worth the complexity. See recent Nesterov work for a plausible implementation.
	\item 0th order: Extremely slow and finding gradient is cheap anyway, usually not worth it.
	\item coordinate descent: heavily depends on the structure.
\end{itemize}

\subsubsection{Attempt 3}
By assumption, $ 0 \preceq \nabla ^2 f(x) \preceq L I$. Thus, for all $ y$,
\[
	\frac{1}{2} \langle y, \nabla ^2 f(x)\ y \rangle \leq \frac{1}{2} L \norm{ y}^2 
.\] 
This allows us to upper bound the quadratic surrogate and simplify it further by removing the Hessian. Notice $ (LI) ^{-1} = \frac{1}{L} I$ which replaces $ (\nabla ^2 f)^{-1}$. So we can modify Newton's method as
\begin{align*}
	x_{k+1} &= \argmin_{x} \underbrace{ f(x_k) + \langle \nabla f(x_k) , x- x_k \rangle + \frac{1}{2} L \norm{ x -x_k}^2}_{q_k(x)}\\ 
		&= x_k - \frac{1}{L} \nabla f(x_k)
\end{align*}
This is $ \mathcal{ O}(n)$. Here $ q_k(x) \geq f(x) \ \forall \ x$ is more than a linearization but is less than the full 2nd order Taylor expansion. It is a \allbold{majorizer} of $ f$. 

fig

\subsubsection{Majorization-minimization (MM)} 
MM can always guarantee making progress on the minimization. The framework is
\begin{enumerate}[label=\arabic*)]
	\item Assume we can always construct a majorizer $ q_k$ s.t.
		\begin{enumerate}[label=(\roman*)]
			\item $ \ \forall \ x, f(x) \leq q_k(x)$
			\item $ f(x_k) = q_k(x_k)$ 
		\end{enumerate}
	\item Iterate: $ x_{k+1} \in \argmin_{x} q_k(x)$.
\end{enumerate}
This algorithm is a \allbold{descent algorithm}. That is, it never makes things worse.
\begin{proof}
\begin{align*}
	f(x_{k+1}) &\leq q_k (x_{k+1}) \qquad  \text{ by (i)} \\
		   &\leq q_k(x_k) \qquad  \text{ by 2)} \\
		   &= f(x_k) \qquad \text{ by (ii)}  
\end{align*}
\end{proof}

Usually we might eventually show that (no convexity needed):
\begin{itemize}
	\item If $ f(x)$ is bounded below, then  $ f(x_k)$ converges by MCT.
	\item If $ (x_k)$ converges and $ f$ is lsc, then the limit  $ x_k \to x$ is a stationary point \emph{i.e.} $ \nabla f(x) =0$.
\end{itemize}
\begin{eg}[usually non-convex]
	~\begin{enumerate}[label=\arabic*)]
		\item Expectation maximization (EM) for maximum-likelihood estimation.
		\item Difference of convex functions (DC) or convex + concave:
			\[
				f(x) = g(x) - h(x)
			,\]
			where $ g,h$ are both convex. Although $ -h$ is concave, $ -h$ is majorized by its tangent line which is convex. Then
			\[
				q_k(x) = g(x) - \underbrace{ (h(x_k) + \langle \nabla h(x_k),x-x_k \rangle)}_{ \text{ affine in }x } 
			\]
			is a majorizer, and $ q_k(x)$ is convex.
	\end{enumerate}
The takeaway is that not all non-convex problems are equally hard.
\end{eg}
\end{document}
