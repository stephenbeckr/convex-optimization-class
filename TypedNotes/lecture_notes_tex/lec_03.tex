\documentclass[class=article,crop=false]{standalone} 
%Fall 2020
% Some basic packages
\usepackage{standalone}[subpreambles=true]
\usepackage[utf8]{inputenc}
\usepackage[T1]{fontenc}
\usepackage{textcomp}
\usepackage[english]{babel}
\usepackage{url}
\usepackage{graphicx}
\usepackage{float}
\usepackage{enumitem}
\usepackage{lmodern}
\usepackage{hyperref}
\usepackage[usenames,svgnames,dvipsnames]{xcolor}


\pdfminorversion=7

% Don't indent paragraphs, leave some space between them
\usepackage{parskip}

% Hide page number when page is empty
\usepackage{emptypage}
\usepackage{subcaption}
\usepackage{multicol}
\usepackage[dvipsnames]{xcolor}
\usepackage[b]{esvect}

% Math stuff
\usepackage{amsmath, amsfonts, mathtools, amsthm, amssymb}
\usepackage{bbm}

% Fancy script capitals
\usepackage{mathrsfs}
\usepackage{cancel}
% Bold math
\usepackage{bm}
% Some shortcuts
\newcommand{\rr}{\ensuremath{\mathbb{R}}}
\newcommand{\zz}{\ensuremath{\mathbb{Z}}}
\newcommand{\qq}{\ensuremath{\mathbb{Q}}}
\newcommand{\nn}{\ensuremath{\mathbb{N}}}
\newcommand{\ff}{\ensuremath{\mathbb{F}}}
\newcommand{\cc}{\ensuremath{\mathbb{C}}}
\newcommand{\ee}{\ensuremath{\mathbb{E}}}
\renewcommand\O{\ensuremath{\emptyset}}
\newcommand{\norm}[1]{{\left\lVert{#1}\right\rVert}}
\newcommand{\ve}[1]{{\mathbf{#1}}}
\newcommand\allbold[1]{{\boldmath\textbf{#1}}}
\DeclareMathOperator{\lcm}{lcm}
\DeclareMathOperator{\im}{im}
\DeclareMathOperator{\coim}{coim}
\DeclareMathOperator{\dom}{dom}
\DeclareMathOperator{\tr}{tr}
\DeclareMathOperator{\rank}{rank}
\DeclareMathOperator*{\var}{Var}
\DeclareMathOperator*{\ev}{E}
\DeclareMathOperator{\sinc}{sinc}
\DeclareMathOperator{\dg}{deg}
\DeclareMathOperator{\aff}{aff}
\DeclareMathOperator{\conv}{conv}
\DeclareMathOperator{\epi}{epi}
\DeclareMathOperator{\inte}{int}
\DeclareMathOperator{\ri}{ri}
\DeclareMathOperator*{\argmin}{argmin}
\DeclareMathOperator*{\argmax}{argmax}
\DeclareMathOperator{\graph}{graph}
\DeclareMathOperator{\sgn}{sgn}
\DeclareMathOperator*{\Rep}{Rep}
\DeclareMathOperator{\Proj}{Proj}
\DeclareMathOperator{\prox}{prox}
\DeclareMathOperator{\mat}{mat}
\let\vec\relax
\DeclareMathOperator{\vec}{vec}
\let\Re\relax
\DeclareMathOperator{\Re}{Re}
\let\Im\relax
\DeclareMathOperator{\Im}{Im}
% Put x \to \infty below \lim
\let\svlim\lim\def\lim{\svlim\limits}

%wide hat
\usepackage{scalerel,stackengine}
\stackMath
\newcommand*\wh[1]{%
\savestack{\tmpbox}{\stretchto{%
  \scaleto{%
    \scalerel*[\widthof{\ensuremath{#1}}]{\kern-.6pt\bigwedge\kern-.6pt}%
    {\rule[-\textheight/2]{1ex}{\textheight}}%WIDTH-LIMITED BIG WEDGE
  }{\textheight}% 
}{0.5ex}}%
\stackon[1pt]{#1}{\tmpbox}%
}
\parskip 1ex

%Make implies and impliedby shorter
\let\implies\Rightarrow
\let\impliedby\Leftarrow
\let\iff\Leftrightarrow
\let\epsilon\varepsilon

% Add \contra symbol to denote contradiction
\usepackage{stmaryrd} % for \lightning
\newcommand\contra{\scalebox{1.5}{$\lightning$}}

% \let\phi\varphi

% Command for short corrections
% Usage: 1+1=\correct{3}{2}

\definecolor{correct}{HTML}{009900}
\newcommand\correct[2]{\ensuremath{\:}{\color{red}{#1}}\ensuremath{\to }{\color{correct}{#2}}\ensuremath{\:}}
\newcommand\green[1]{{\color{correct}{#1}}}

% horizontal rule
\newcommand\hr{
    \noindent\rule[0.5ex]{\linewidth}{0.5pt}
}

% hide parts
\newcommand\hide[1]{}

% si unitx
\usepackage{siunitx}
\sisetup{locale = FR}

%allows pmatrix to stretch
\makeatletter
\renewcommand*\env@matrix[1][\arraystretch]{%
  \edef\arraystretch{#1}%
  \hskip -\arraycolsep
  \let\@ifnextchar\new@ifnextchar
  \array{*\c@MaxMatrixCols c}}
\makeatother

\renewcommand{\arraystretch}{0.8}

% Environments
\makeatother
% For box around Definition, Theorem, \ldots
%%fakesection Theorems
\usepackage{thmtools}
\usepackage[framemethod=TikZ]{mdframed}

\theoremstyle{definition}
\mdfdefinestyle{mdbluebox}{%
	roundcorner = 10pt,
	linewidth=1pt,
	skipabove=12pt,
	innerbottommargin=9pt,
	skipbelow=2pt,
	nobreak=true,
	linecolor=blue,
	backgroundcolor=TealBlue!5,
}
\declaretheoremstyle[
	headfont=\sffamily\bfseries\color{MidnightBlue},
	mdframed={style=mdbluebox},
	headpunct={\\[3pt]},
	postheadspace={0pt}
]{thmbluebox}

\mdfdefinestyle{mdredbox}{%
	linewidth=0.5pt,
	skipabove=12pt,
	frametitleaboveskip=5pt,
	frametitlebelowskip=0pt,
	skipbelow=2pt,
	frametitlefont=\bfseries,
	innertopmargin=4pt,
	innerbottommargin=8pt,
	nobreak=false,
	linecolor=RawSienna,
	backgroundcolor=Salmon!5,
}
\declaretheoremstyle[
	headfont=\bfseries\color{RawSienna},
	mdframed={style=mdredbox},
	headpunct={\\[3pt]},
	postheadspace={0pt},
]{thmredbox}

\declaretheorem[%
style=thmbluebox,name=Theorem,numberwithin=section]{thm}
\declaretheorem[style=thmbluebox,name=Lemma,sibling=thm]{lem}
\declaretheorem[style=thmbluebox,name=Proposition,sibling=thm]{prop}
\declaretheorem[style=thmbluebox,name=Corollary,sibling=thm]{coro}
\declaretheorem[style=thmredbox,name=Example,sibling=thm]{eg}

\mdfdefinestyle{mdgreenbox}{%
	roundcorner = 10pt,
	linewidth=1pt,
	skipabove=12pt,
	innerbottommargin=9pt,
	skipbelow=2pt,
	nobreak=true,
	linecolor=ForestGreen,
	backgroundcolor=ForestGreen!5,
}

\declaretheoremstyle[
	headfont=\bfseries\sffamily\color{ForestGreen!70!black},
	bodyfont=\normalfont,
	spaceabove=2pt,
	spacebelow=1pt,
	mdframed={style=mdgreenbox},
	headpunct={ --- },
]{thmgreenbox}

\declaretheorem[style=thmgreenbox,name=Definition,sibling=thm]{defn}

\mdfdefinestyle{mdgreenboxsq}{%
	linewidth=1pt,
	skipabove=12pt,
	innerbottommargin=9pt,
	skipbelow=2pt,
	nobreak=true,
	linecolor=ForestGreen,
	backgroundcolor=ForestGreen!5,
}
\declaretheoremstyle[
	headfont=\bfseries\sffamily\color{ForestGreen!70!black},
	bodyfont=\normalfont,
	spaceabove=2pt,
	spacebelow=1pt,
	mdframed={style=mdgreenboxsq},
	headpunct={},
]{thmgreenboxsq}
\declaretheoremstyle[
	headfont=\bfseries\sffamily\color{ForestGreen!70!black},
	bodyfont=\normalfont,
	spaceabove=2pt,
	spacebelow=1pt,
	mdframed={style=mdgreenboxsq},
	headpunct={},
]{thmgreenboxsq*}

\mdfdefinestyle{mdblackbox}{%
	skipabove=8pt,
	linewidth=3pt,
	rightline=false,
	leftline=true,
	topline=false,
	bottomline=false,
	linecolor=black,
	backgroundcolor=RedViolet!5!gray!5,
}
\declaretheoremstyle[
	headfont=\bfseries,
	bodyfont=\normalfont\small,
	spaceabove=0pt,
	spacebelow=0pt,
	mdframed={style=mdblackbox}
]{thmblackbox}

\theoremstyle{plain}
\declaretheorem[name=Question,sibling=thm,style=thmblackbox]{ques}
\declaretheorem[name=Remark,sibling=thm,style=thmgreenboxsq]{remark}
\declaretheorem[name=Remark,sibling=thm,style=thmgreenboxsq*]{remark*}

\theoremstyle{definition}
\newtheorem{claim}[thm]{Claim}
\theoremstyle{remark}
\newtheorem*{case}{Case}
\newtheorem*{notation}{Notation}
\newtheorem*{note}{Note}
\newtheorem*{motivation}{Motivation}
\newtheorem*{intuition}{Intuition}

% Make section starts with 1 for report type
%\renewcommand\thesection{\arabic{section}}

% End example and intermezzo environments with a small diamond (just like proof
% environments end with a small square)
\usepackage{etoolbox}
\AtEndEnvironment{vb}{\null\hfill$\diamond$}%
\AtEndEnvironment{intermezzo}{\null\hfill$\diamond$}%
% \AtEndEnvironment{opmerking}{\null\hfill$\diamond$}%

% Fix some spacing
% http://tex.stackexchange.com/questions/22119/how-can-i-change-the-spacing-before-theorems-with-amsthm
\makeatletter
\def\thm@space@setup{%
  \thm@preskip=\parskip \thm@postskip=0pt
}

% Fix some stuff
% %http://tex.stackexchange.com/questions/76273/multiple-pdfs-with-page-group-included-in-a-single-page-warning
\pdfsuppresswarningpagegroup=1

\renewcommand{\baselinestretch}{1.5}
\RequirePackage{hyperref}[6.83]
\hypersetup{
  colorlinks=false,
  frenchlinks=false,
  pdfborder={0 0 0},
  naturalnames=false,
  hypertexnames=false,
  breaklinks
}
\urlstyle{same}

\usepackage{graphics}
\usepackage{epstopdf}

%%
%% Add support for color in order to color the hyperlinks.
%% 
\hypersetup{
  colorlinks = true,
  allcolors = siaminlinkcolor,
  urlcolor = siamexlinkcolor,
}
%%fakesection Links
\hypersetup{
    colorlinks,
    linkcolor={red!50!black},
    citecolor={green!50!black},
    urlcolor={blue!80!black}
}
%customization of cleveref
\RequirePackage[capitalize,nameinlink]{cleveref}[0.19]

% Per SIAM Style Manual, "section" should be lowercase
\crefname{section}{section}{sections}
\crefname{subsection}{subsection}{subsections}
\Crefname{section}{Section}{Sections}
\Crefname{subsection}{Subsection}{Subsections}

% Per SIAM Style Manual, "Figure" should be spelled out in references
\Crefname{figure}{Figure}{Figures}

% Per SIAM Style Manual, don't say equation in front on an equation.
\crefformat{equation}{\textup{#2(#1)#3}}
\crefrangeformat{equation}{\textup{#3(#1)#4--#5(#2)#6}}
\crefmultiformat{equation}{\textup{#2(#1)#3}}{ and \textup{#2(#1)#3}}
{, \textup{#2(#1)#3}}{, and \textup{#2(#1)#3}}
\crefrangemultiformat{equation}{\textup{#3(#1)#4--#5(#2)#6}}%
{ and \textup{#3(#1)#4--#5(#2)#6}}{, \textup{#3(#1)#4--#5(#2)#6}}{, and \textup{#3(#1)#4--#5(#2)#6}}

% But spell it out at the beginning of a sentence.
\Crefformat{equation}{#2Equation~\textup{(#1)}#3}
\Crefrangeformat{equation}{Equations~\textup{#3(#1)#4--#5(#2)#6}}
\Crefmultiformat{equation}{Equations~\textup{#2(#1)#3}}{ and \textup{#2(#1)#3}}
{, \textup{#2(#1)#3}}{, and \textup{#2(#1)#3}}
\Crefrangemultiformat{equation}{Equations~\textup{#3(#1)#4--#5(#2)#6}}%
{ and \textup{#3(#1)#4--#5(#2)#6}}{, \textup{#3(#1)#4--#5(#2)#6}}{, and \textup{#3(#1)#4--#5(#2)#6}}

% Make number non-italic in any environment.
\crefdefaultlabelformat{#2\textup{#1}#3}

% My name
\author{Jaden Wang}



\begin{document}

\subsection{Convexity}
\begin{note}
	From now on we always assume the constraint set $ C$ is a subset of a vector space.
\end{note}
\begin{defn}[convex set]
	A set $ C$ is  \allbold{convex} if for all $ x,y \in C$ and for all $ t \in [0,1]$, then
	\[
		tx + (1-t)y \in C
	.\] 
\end{defn}

~\begin{figure}[H]
	\centering
	\includegraphics[width=0.9\textwidth]{./figures/cvx_set.jpg}
\end{figure}
~\begin{defn}[convex function]
	$ f : \rr^{n} \to \rr$ is a \allbold{convex function} if for all $ x,y \in \rr^{n}$ and $ t \in [0,1]$, then
	\[
		f(t x+ (1-t)y) \leq t f(x) + (1-t) f(y)
	.\] 
\end{defn}
~\begin{figure}[H]
	\centering
	\includegraphics[width=1.0\textwidth]{./figures/cvx_fcn.jpg}
\end{figure}

\begin{remark}
Linear or affine functions are both convex and concave.
\end{remark}

Recall that a \allbold{graph} of a function is just the set of points we use to plot a function (now generalized to functions with domain of any dimension).

\begin{defn}[epigraph]
\[
	\epi(f) = \{(x,s): x \in \rr^{n}, s \in \rr, s\geq f(x)\}  
.\] 
\end{defn}
\begin{intuition}
The epigraph of $ f$ is sort of the "upper" partition of the vector space that the graph of $ f$ resides and partitions. We use epigraph to bridge the concepts of convex sets and convex functions.
\end{intuition}

~\begin{figure}[H]
	\centering
	\includegraphics[width=0.8\textwidth]{./figures/epi.jpg}
\end{figure}

\begin{prop}
	A function $ f$ is convex if and only if  the set $ \epi(f)$ is convex.
\end{prop}

\begin{thm}
	If $ f$ is convex and  $ C$ is convex, then any local minimizer of  $ \min_{x \in C } f(x)$ is in fact global. The set of global solutions is also convex and in particular connected.
\end{thm}
This is very neat for optimization!
\begin{proof}
	Given a local minimizer $ x^* \in C$, let's show that it is a global minimizer. Suppose not, that is, there exists a point $ x \in C $ s.t. $ f(x) < f(x^* )$. Since $ x^* $ is a local minimizer, there exists an $ \epsilon>0$ s.t. $ f(x^* ) \leq f(y)$ for all $ y \in C$ s.t. $\norm{ y -x^* }< \epsilon $. Clearly $ \norm{ x^* -x} \geq \epsilon$ or $ x^* $ would not be a local minimizer. Choose $ t < \frac{ \epsilon}{ \norm{ x^* -x } } \in [0,1]$. Since $ C$ is convex, we know that the point $ x_0 = t x + (1-t)x^*  \in C$. Notice that
\begin{align*}
	\norm{ x^* -x_0} &=\norm{ x^* - (t x+ ( 1-t) x^* )  }\\
			 &= \norm{ t (x^* - x) }  \\
					 &= t \norm{ x^* -x}   \\
					 &< t \cdot \frac{ \epsilon}{ t } \\
					 &= \epsilon 
\end{align*}
That is, $ x_0$ in the $ \epsilon$-neighborhood of $ x^* $ and it follows that $ f(x^* ) \leq f(x_0)$. 
	Since $ f$ is convex,
	\begin{align*}
		f(x_0) &= f(t x + (1-t) x^* ) \\
		       &\leq t f(x) + (1-t) f(x^* )\\
		       &< tf(x^* ) + (1-t) f(x^* )\\
		       &= f(x^* )
	\end{align*}
	This contradicts with the fact that $ f(x^* ) \leq f(x_0)$. Hence we prove that any local minimizer $ x^* $ must also be a global minimizer.

	To show that the set of global minimizers is connected, it suffices to prove that it is path-connected. The path we check is of course the line segment in the definition of convex set:
\[
	g(t) = t a + (1-t) b, \qquad a, b \in \argmin_{x \in C} f(x)
.\] 
It's easy to see that $ f(g(t)) \leq tf(a)+(1-t)f(b) = \min_{x \in C} f(x)$ for all $ t \in [0,1]$. It follows that $ f(g(t))$ must equal to the global minimum for all $ t \in [0,1]$. This makes $ g(t) \in \argmin_{x \in C} f(x) \ \forall \ t \in [0,1]$. Thus, the continuous function $ g(t)$ is the path we seek. Since $ a,b$ are arbitrary global minimizers, we kill two birds in one stone and show that the set of global minimizers is 1. convex and 2. path-connected and therefore connected.
\end{proof}
\newpage

\section{Convex Sets [BV04 Ch.2] }
\subsection{Convex, affine, and cone}
~\begin{defn}
	Let $ x,y \in \rr^{n}$ (or any vector space), then
	\begin{enumerate}[label=\arabic*)]
		\item $ tx + (1-t)y, t \in [0,1]$ is a \allbold{convex combination} (of $ x,y$).
		\item $ t x+(1-t)y, t \in \rr$ is a \allbold{linear combination}.
		\item $ t x + sy, t,s \geq 0$ is a  \allbold{(convex) conic combination}. 
	\end{enumerate}
\end{defn}

\begin{defn}
A set $ C \subseteq \rr^{n}$ is
\begin{enumerate}[label=\arabic*)]
	\item \allbold{convex} if for all $ x,y \in C$, it contains all convex combinations of $ x,y$.
	\item  \allbold{affine} if for all $ x,y \in C$, it contains all linear combinations of $ x,y$.
	\item a  \allbold{cone} if for all $ x \in C$, it contains all conic combinations of $ x$.
	\item a  \allbold{convex cone} if it's convex and a cone. That is, for all $ x,y \in C, t,x\geq 0$, $ tx + sy \in C$. 
\end{enumerate}
\end{defn}
\begin{note}
Affine implies convex based on definition.
\end{note}

\begin{remark}
	An affine set/subspace is like a subspace except it is possible shifted (may not include 0). Think inhomogenous equation from differential equations. It's also analogous to cosets.
\end{remark}

Recall from analysis, the \allbold{closure} of $ A$, $ \overline{A}$, is the union of $ A$ and all its limit points. We can also characterize $ \overline{A}$ as the smallest closed set containing $ A$ or equivalently the intersection of all closed sets containing $ A$. We can do something similar here.

\begin{defn}[affine hull]
	 The \allbold{affine hull} of $ C$,  $ \aff(C)$, is the smallest affine set containing  $ C$.\end{defn}
	 The \allbold{affine dimension} of $ C$ is  $ \dim(\aff (C))$. For example, although the unit circle in $ \rr^{2}$ has dimension 1, its affine hull is all of $ \rr^2$ so its affine dimension is 2.

\begin{defn}[convex hull]
	The \allbold{convex hull} of $ C$,  $ \conv(C)$, is the smallest convex set containing  $ C$. It is equivalent to the set of all convex combinations of points in $ C$:
	 \[
	 \left\{\sum_{ i= 1}^{ k} t_i x_i: x_i \in C, t_i \geq 0, \sum_{ i= 1}^{ k} t_i = 1 \right\} 
	 .\] 

\end{defn}
\begin{intuition}
	Given an arbitrary set $ C$, we wrap a rubber band around it and the region enclosed by the rubber band is $ \conv(C)$.
\end{intuition}
\begin{figure}[H]
	\centering
	\includegraphics[width=0.4\textwidth]{./figures/cvx_hull.png}
\end{figure}

\begin{defn}[conic hull]
The \allbold{conic hull} of $ C$ is the set of all conic combinations of points in  $ C$. That is,
	 \[
	 \left\{\sum_{ i= 1}^{ k} t_i x_i: x_i \in C, t_i \geq 0 \right\} 
	 .\]
	 It is the smallest convex cone that contains $ C$.
\end{defn}
\begin{defn}[relative interior]
	The \allbold{relative interior} of a set $ C$ is the set
	\[ \ri(C) = \{x \in C: \ \exists \ \epsilon>0 , B_{ \epsilon} ( x ) \cap \aff(C) \subseteq C\} \]
\end{defn}

\begin{note}
This is really useful for studying symmetric matrices. 
\end{note}

\begin{eg}
	Let $ C = [0,1] \subseteq \rr$, then $ \inte (C) = \ri(C) = (0,1)$.

	However, if $ C = [0,1] \times \{0\} $ which has the same shape but is a embedding in $ \rr^2$, then $ \inte (C) = \O$ because for every $ x \in C$, $ B_{ \epsilon} ( x )$ goes outside $ C$ along the second dimension. But $ \ri (C) = (0,1)$ because  $ \aff(C) = \rr \times \{0\} \cong \rr $.
\end{eg}

\subsection{Important examples}
~\begin{defn}[hyperplane]
	For $  a \in \rr^{n}, b \in \rr$, the \allbold{hyperplane} would be the affine, $ n-1$ dimensional set
 \[
\{x \in \rr^{n}: a^{T} x =b\} 
.\] 
Alternatively,
\[
	\{x \in \rr^{n}: a^{T} (x - x_0) =0\} 
.\] 
\end{defn}
\begin{note}
Hyperplanes are convex and affine with $ n-1$ dimension. In 2D, it's a line. In 3D it's an actual plane. Also recall from cal 3 that $ a$ is the normal vector of the hyperplane.
\end{note}

\begin{defn}[half-space]
A hyperplane partition $ \rr^{n}$ into two \allbold{half-spaces}. They have the form 
\[
	\{x \in \rr^{n}: a^{T} x \leq b\}
.\] 
\end{defn}
\begin{note}
Half spaces are convex but not affine.
\end{note}
\begin{defn}[Euclidean ball]
Open ball: $ B_{ \epsilon} ( x ) = \{y \in \rr^{n}: \norm{ y-x} < \epsilon\} $.

Closed ball: $ \overline{ B_{ \epsilon} } (x) = \{ y \in \rr^{n}, \norm{ y- x} \leq \epsilon\} $.
\end{defn}
\begin{note}
Balls are convex but not affine.
\end{note}
\newpage
\begin{defn}[ellipsoid]
	An \allbold{ellipsoid} has the form 
\[
	\mathcal{ E} = \{x: (x-x_0)^{T} P^{-1} (x-x_0) \leq 1\} 
\]
for some matrix $ P \succ 0$.
\end{defn}
\begin{notation}
	$A \succ 0$ in this course means $ A$ is symmetric and positive definite. 
\end{notation}
\begin{note}
Ellipsoid, like ball, is convex but not affine. If we choose $ P = \epsilon^2 I$, then we get an $ \epsilon$-ball.
\end{note}
\begin{intuition}
	This is a generalization of Cal 3 ellipsoid using quadratic form. Recall that the $P^{-1}$ in the middle of $ x^{T} x$ is giving us a \emph{weighted} sum. Since $ y^{T} y = \norm{ y}^2  \leq 1$ is a unit ball, a weighted norm would help us transform an ellipsoid into the unit ball. We use the inverse of $ P$ in the definition because the image of this quadratic form is sort of a unit ball, but we are more interested in knowing how to go from the unit ball to the ellipsoid, and $ P$ encodes this transformation.  Also since $ P \succ 0$, using Spectral Theorem we can find the principle axises and length of the ellipsoid.
\end{intuition}


\begin{eg}[cones]
	~\begin{itemize}
		\item positive orthant in $ \rr^{n}$, \emph{e.g.} first quadrant in $ \rr^2$.

			$ \rr^{n}_+ = \{x \in \rr^{n}: x_i \geq 0\} $ is a cone.

			However, $ \rr^{n}_{++} = \{x \in \rr^{n}: x_i >0\} $ is not a cone as a cone must include the additive identity 0 to be closed under non-negative scalar multiplication.
		\item Lorentz cone/2nd order cone/"ice cream cone"
			\[
				C= \{(x,t) \in \rr^{n+1}, x \in \rr^{n}, t \in \rr: \norm{ x}_2 \leq t   \} 
			.\]
		\item the set of positive semidefinite matrices (PSD): this is the most important nonployhedral cone. We assume PSDs are Hermitian and are denoted by $A \succeq 0$.
			\begin{notation}
				$ \mathbb{S}^n$ denotes the set of symmetric  $ n\times n$ matrices. Similar to the reals, we use $ \mathbb{S}_+^n$ to denote the set of symmetric positive semidefinite matrices. 
			\end{notation}
	\end{itemize}
\end{eg}

\begin{defn}[polyhedron]
	A \allbold{polyhedron} $ \mathcal{ P} \subseteq \rr^{n}$ is a set of the intersection of a \emph{finite} number of half-spaces and hyperplanes. That is,
	\[
	\mathcal{ P} = \{x \in \rr^{n}: a_j^{T} x \leq b_j, j=1,\ldots,m; c_j^{T}x= d_j, j=1,\ldots,p\} 
	.\] 
\end{defn}

\begin{note}
	Intersection of infinite number of half-spaces and hyperplanes are not necessarily a polyhedron (in the intuitive sense) because it can "smooth out" the edges and turn it into for example a ball, such as
\[
	\overline{B_{ 1}} (  0) = \{x \in \rr^{n}: a_j^{T} x \leq 1, a_j \in \text{ unit circle} \} 
.\] 
\end{note}
\begin{note}
	Polyhedra are always convex since it's finite intersection of convex sets.
\end{note}

\begin{note}
The terms "polygon", "polyhedron", and "polytope" will be used interchangeably in this course.
\end{note}
\end{document}
