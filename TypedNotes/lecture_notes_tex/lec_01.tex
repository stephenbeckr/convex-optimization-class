\documentclass[class=article,crop=false]{standalone} 
%Fall 2020
% Some basic packages
\usepackage{standalone}[subpreambles=true]
\usepackage[utf8]{inputenc}
\usepackage[T1]{fontenc}
\usepackage{textcomp}
\usepackage[english]{babel}
\usepackage{url}
\usepackage{graphicx}
\usepackage{float}
\usepackage{enumitem}
\usepackage{lmodern}
\usepackage{hyperref}
\usepackage[usenames,svgnames,dvipsnames]{xcolor}


\pdfminorversion=7

% Don't indent paragraphs, leave some space between them
\usepackage{parskip}

% Hide page number when page is empty
\usepackage{emptypage}
\usepackage{subcaption}
\usepackage{multicol}
\usepackage[dvipsnames]{xcolor}
\usepackage[b]{esvect}

% Math stuff
\usepackage{amsmath, amsfonts, mathtools, amsthm, amssymb}
\usepackage{bbm}

% Fancy script capitals
\usepackage{mathrsfs}
\usepackage{cancel}
% Bold math
\usepackage{bm}
% Some shortcuts
\newcommand{\rr}{\ensuremath{\mathbb{R}}}
\newcommand{\zz}{\ensuremath{\mathbb{Z}}}
\newcommand{\qq}{\ensuremath{\mathbb{Q}}}
\newcommand{\nn}{\ensuremath{\mathbb{N}}}
\newcommand{\ff}{\ensuremath{\mathbb{F}}}
\newcommand{\cc}{\ensuremath{\mathbb{C}}}
\newcommand{\ee}{\ensuremath{\mathbb{E}}}
\renewcommand\O{\ensuremath{\emptyset}}
\newcommand{\norm}[1]{{\left\lVert{#1}\right\rVert}}
\newcommand{\ve}[1]{{\mathbf{#1}}}
\newcommand\allbold[1]{{\boldmath\textbf{#1}}}
\DeclareMathOperator{\lcm}{lcm}
\DeclareMathOperator{\im}{im}
\DeclareMathOperator{\coim}{coim}
\DeclareMathOperator{\dom}{dom}
\DeclareMathOperator{\tr}{tr}
\DeclareMathOperator{\rank}{rank}
\DeclareMathOperator*{\var}{Var}
\DeclareMathOperator*{\ev}{E}
\DeclareMathOperator{\sinc}{sinc}
\DeclareMathOperator{\dg}{deg}
\DeclareMathOperator{\aff}{aff}
\DeclareMathOperator{\conv}{conv}
\DeclareMathOperator{\epi}{epi}
\DeclareMathOperator{\inte}{int}
\DeclareMathOperator{\ri}{ri}
\DeclareMathOperator*{\argmin}{argmin}
\DeclareMathOperator*{\argmax}{argmax}
\DeclareMathOperator{\graph}{graph}
\DeclareMathOperator{\sgn}{sgn}
\DeclareMathOperator*{\Rep}{Rep}
\DeclareMathOperator{\Proj}{Proj}
\DeclareMathOperator{\prox}{prox}
\DeclareMathOperator{\mat}{mat}
\let\vec\relax
\DeclareMathOperator{\vec}{vec}
\let\Re\relax
\DeclareMathOperator{\Re}{Re}
\let\Im\relax
\DeclareMathOperator{\Im}{Im}
% Put x \to \infty below \lim
\let\svlim\lim\def\lim{\svlim\limits}

%wide hat
\usepackage{scalerel,stackengine}
\stackMath
\newcommand*\wh[1]{%
\savestack{\tmpbox}{\stretchto{%
  \scaleto{%
    \scalerel*[\widthof{\ensuremath{#1}}]{\kern-.6pt\bigwedge\kern-.6pt}%
    {\rule[-\textheight/2]{1ex}{\textheight}}%WIDTH-LIMITED BIG WEDGE
  }{\textheight}% 
}{0.5ex}}%
\stackon[1pt]{#1}{\tmpbox}%
}
\parskip 1ex

%Make implies and impliedby shorter
\let\implies\Rightarrow
\let\impliedby\Leftarrow
\let\iff\Leftrightarrow
\let\epsilon\varepsilon

% Add \contra symbol to denote contradiction
\usepackage{stmaryrd} % for \lightning
\newcommand\contra{\scalebox{1.5}{$\lightning$}}

% \let\phi\varphi

% Command for short corrections
% Usage: 1+1=\correct{3}{2}

\definecolor{correct}{HTML}{009900}
\newcommand\correct[2]{\ensuremath{\:}{\color{red}{#1}}\ensuremath{\to }{\color{correct}{#2}}\ensuremath{\:}}
\newcommand\green[1]{{\color{correct}{#1}}}

% horizontal rule
\newcommand\hr{
    \noindent\rule[0.5ex]{\linewidth}{0.5pt}
}

% hide parts
\newcommand\hide[1]{}

% si unitx
\usepackage{siunitx}
\sisetup{locale = FR}

%allows pmatrix to stretch
\makeatletter
\renewcommand*\env@matrix[1][\arraystretch]{%
  \edef\arraystretch{#1}%
  \hskip -\arraycolsep
  \let\@ifnextchar\new@ifnextchar
  \array{*\c@MaxMatrixCols c}}
\makeatother

\renewcommand{\arraystretch}{0.8}

% Environments
\makeatother
% For box around Definition, Theorem, \ldots
%%fakesection Theorems
\usepackage{thmtools}
\usepackage[framemethod=TikZ]{mdframed}

\theoremstyle{definition}
\mdfdefinestyle{mdbluebox}{%
	roundcorner = 10pt,
	linewidth=1pt,
	skipabove=12pt,
	innerbottommargin=9pt,
	skipbelow=2pt,
	nobreak=true,
	linecolor=blue,
	backgroundcolor=TealBlue!5,
}
\declaretheoremstyle[
	headfont=\sffamily\bfseries\color{MidnightBlue},
	mdframed={style=mdbluebox},
	headpunct={\\[3pt]},
	postheadspace={0pt}
]{thmbluebox}

\mdfdefinestyle{mdredbox}{%
	linewidth=0.5pt,
	skipabove=12pt,
	frametitleaboveskip=5pt,
	frametitlebelowskip=0pt,
	skipbelow=2pt,
	frametitlefont=\bfseries,
	innertopmargin=4pt,
	innerbottommargin=8pt,
	nobreak=false,
	linecolor=RawSienna,
	backgroundcolor=Salmon!5,
}
\declaretheoremstyle[
	headfont=\bfseries\color{RawSienna},
	mdframed={style=mdredbox},
	headpunct={\\[3pt]},
	postheadspace={0pt},
]{thmredbox}

\declaretheorem[%
style=thmbluebox,name=Theorem,numberwithin=section]{thm}
\declaretheorem[style=thmbluebox,name=Lemma,sibling=thm]{lem}
\declaretheorem[style=thmbluebox,name=Proposition,sibling=thm]{prop}
\declaretheorem[style=thmbluebox,name=Corollary,sibling=thm]{coro}
\declaretheorem[style=thmredbox,name=Example,sibling=thm]{eg}

\mdfdefinestyle{mdgreenbox}{%
	roundcorner = 10pt,
	linewidth=1pt,
	skipabove=12pt,
	innerbottommargin=9pt,
	skipbelow=2pt,
	nobreak=true,
	linecolor=ForestGreen,
	backgroundcolor=ForestGreen!5,
}

\declaretheoremstyle[
	headfont=\bfseries\sffamily\color{ForestGreen!70!black},
	bodyfont=\normalfont,
	spaceabove=2pt,
	spacebelow=1pt,
	mdframed={style=mdgreenbox},
	headpunct={ --- },
]{thmgreenbox}

\declaretheorem[style=thmgreenbox,name=Definition,sibling=thm]{defn}

\mdfdefinestyle{mdgreenboxsq}{%
	linewidth=1pt,
	skipabove=12pt,
	innerbottommargin=9pt,
	skipbelow=2pt,
	nobreak=true,
	linecolor=ForestGreen,
	backgroundcolor=ForestGreen!5,
}
\declaretheoremstyle[
	headfont=\bfseries\sffamily\color{ForestGreen!70!black},
	bodyfont=\normalfont,
	spaceabove=2pt,
	spacebelow=1pt,
	mdframed={style=mdgreenboxsq},
	headpunct={},
]{thmgreenboxsq}
\declaretheoremstyle[
	headfont=\bfseries\sffamily\color{ForestGreen!70!black},
	bodyfont=\normalfont,
	spaceabove=2pt,
	spacebelow=1pt,
	mdframed={style=mdgreenboxsq},
	headpunct={},
]{thmgreenboxsq*}

\mdfdefinestyle{mdblackbox}{%
	skipabove=8pt,
	linewidth=3pt,
	rightline=false,
	leftline=true,
	topline=false,
	bottomline=false,
	linecolor=black,
	backgroundcolor=RedViolet!5!gray!5,
}
\declaretheoremstyle[
	headfont=\bfseries,
	bodyfont=\normalfont\small,
	spaceabove=0pt,
	spacebelow=0pt,
	mdframed={style=mdblackbox}
]{thmblackbox}

\theoremstyle{plain}
\declaretheorem[name=Question,sibling=thm,style=thmblackbox]{ques}
\declaretheorem[name=Remark,sibling=thm,style=thmgreenboxsq]{remark}
\declaretheorem[name=Remark,sibling=thm,style=thmgreenboxsq*]{remark*}

\theoremstyle{definition}
\newtheorem{claim}[thm]{Claim}
\theoremstyle{remark}
\newtheorem*{case}{Case}
\newtheorem*{notation}{Notation}
\newtheorem*{note}{Note}
\newtheorem*{motivation}{Motivation}
\newtheorem*{intuition}{Intuition}

% Make section starts with 1 for report type
%\renewcommand\thesection{\arabic{section}}

% End example and intermezzo environments with a small diamond (just like proof
% environments end with a small square)
\usepackage{etoolbox}
\AtEndEnvironment{vb}{\null\hfill$\diamond$}%
\AtEndEnvironment{intermezzo}{\null\hfill$\diamond$}%
% \AtEndEnvironment{opmerking}{\null\hfill$\diamond$}%

% Fix some spacing
% http://tex.stackexchange.com/questions/22119/how-can-i-change-the-spacing-before-theorems-with-amsthm
\makeatletter
\def\thm@space@setup{%
  \thm@preskip=\parskip \thm@postskip=0pt
}

% Fix some stuff
% %http://tex.stackexchange.com/questions/76273/multiple-pdfs-with-page-group-included-in-a-single-page-warning
\pdfsuppresswarningpagegroup=1

\renewcommand{\baselinestretch}{1.5}
\RequirePackage{hyperref}[6.83]
\hypersetup{
  colorlinks=false,
  frenchlinks=false,
  pdfborder={0 0 0},
  naturalnames=false,
  hypertexnames=false,
  breaklinks
}
\urlstyle{same}

\usepackage{graphics}
\usepackage{epstopdf}

%%
%% Add support for color in order to color the hyperlinks.
%% 
\hypersetup{
  colorlinks = true,
  allcolors = siaminlinkcolor,
  urlcolor = siamexlinkcolor,
}
%%fakesection Links
\hypersetup{
    colorlinks,
    linkcolor={red!50!black},
    citecolor={green!50!black},
    urlcolor={blue!80!black}
}
%customization of cleveref
\RequirePackage[capitalize,nameinlink]{cleveref}[0.19]

% Per SIAM Style Manual, "section" should be lowercase
\crefname{section}{section}{sections}
\crefname{subsection}{subsection}{subsections}
\Crefname{section}{Section}{Sections}
\Crefname{subsection}{Subsection}{Subsections}

% Per SIAM Style Manual, "Figure" should be spelled out in references
\Crefname{figure}{Figure}{Figures}

% Per SIAM Style Manual, don't say equation in front on an equation.
\crefformat{equation}{\textup{#2(#1)#3}}
\crefrangeformat{equation}{\textup{#3(#1)#4--#5(#2)#6}}
\crefmultiformat{equation}{\textup{#2(#1)#3}}{ and \textup{#2(#1)#3}}
{, \textup{#2(#1)#3}}{, and \textup{#2(#1)#3}}
\crefrangemultiformat{equation}{\textup{#3(#1)#4--#5(#2)#6}}%
{ and \textup{#3(#1)#4--#5(#2)#6}}{, \textup{#3(#1)#4--#5(#2)#6}}{, and \textup{#3(#1)#4--#5(#2)#6}}

% But spell it out at the beginning of a sentence.
\Crefformat{equation}{#2Equation~\textup{(#1)}#3}
\Crefrangeformat{equation}{Equations~\textup{#3(#1)#4--#5(#2)#6}}
\Crefmultiformat{equation}{Equations~\textup{#2(#1)#3}}{ and \textup{#2(#1)#3}}
{, \textup{#2(#1)#3}}{, and \textup{#2(#1)#3}}
\Crefrangemultiformat{equation}{Equations~\textup{#3(#1)#4--#5(#2)#6}}%
{ and \textup{#3(#1)#4--#5(#2)#6}}{, \textup{#3(#1)#4--#5(#2)#6}}{, and \textup{#3(#1)#4--#5(#2)#6}}

% Make number non-italic in any environment.
\crefdefaultlabelformat{#2\textup{#1}#3}

% My name
\author{Jaden Wang}



\begin{document}
\chapter{Theoretical Foundation}
\newpage
\section{Introduction}
An optimization problem looks like
\[
	\min_{x \in C} f(x)
\]
where $ f(x)$ is the  \allbold{objective function} and $ C \subseteq \rr^n$ is the \allbold{constraint set}. $ C$ might look like
 \[
	 C=\{x: g_i(x) \leq 0 \ \forall \ i=1,\ldots,m\} 
.\] 

\begin{remark}
We can always turn a maximization problem into a minimization problem as the following:
\[
	\min_x f(x) = -\max_x -f(x)
.\] 
Therefore, WLOG, we will stick with minimization.

\end{remark}
	
\begin{eg}
	An assistant professor earns \$100 per day, and they enjoy both ice cream and cake. The optimization problem aims to maximize the utility ( \emph{e.g.} happiness) of ice cream $ f_1(x_1)$ and of cake $ f_2(x_2)$. The constraints we have is that $ x_1\geq 0, x_2 \geq 0$, and $ x_1+x_2 \leq 100$.

	To maximize both utility, it might be natural to define
	\[
		F(\vec{x}) = \begin{pmatrix} f_1(x_1)\\f_2(x_2) \end{pmatrix}, \vec{ x} = \begin{pmatrix} x_1\\x_2 \end{pmatrix}  
	\]
and maximize $ F$. However, this isn't a well-defined problem, because  \emph{there is no total order on $ \rr^n$}! That is, we don't have a good way to compare whether a vector is bigger than another vector, except in the cases when the same direction of inequality can be achieved for all components of two vectors and a partial order can be established. For this kind of \allbold{multi-objective} optimization problem, we can look for Pareto-optimal points in these special cases. We can also try to convert the output into a scalar as the following:
\[
	\min_x f_1(x) + \lambda \cdot  f_2(x_2)
\]
for some $ \lambda>0$ that reflects our preference for cake vs ice cream. But this can be subjective.

\end{eg}


Thus, For the remainder of this class, we are only going to assume $ f: \rr^n \to \rr$. 
\\

Moreover, for $ f: \rr \to \rr$, it's very easy to solve by using root finding algorithms or grid search. So since interesting problems occur with vector inputs, we will simply use $ x$ to represent vectors.

\begin{notation}
	$ \min$ asks for the minimum value, whereas $ \arg\min$ asks for the minimizer that yields the minimum value.
\end{notation}
\newpage
\subsection{Lipschitz continuity}
\begin{eg}
Let's consider a variant of the Dirichlet function, $ f: \rr \to \rr$
\begin{equation*}
	f(x)=
\begin{cases}
	x & \text{ if } x \in \qq\\  
	1 & \text{ if } x \in \rr \setminus \qq 
\end{cases}
\end{equation*}
Then the solution to the problem
\[
	\min_{x \in [0,1]} f(x) = 0
\] 
is $ x=0$ by observation. However, the function is not smooth and a small perturbation can yield wildly different values. Thus, it is not tractable to solve this numerically.
\end{eg}

This requires us to add a smoothness assumption:
\begin{defn}
	$ f: \rr^{n} \to \rr$ is \allbold{$L$-Lipschitz continuous} with respect to a norm $ \norm{ \cdot } $ if for all $ x, y \in \rr^{n}$,
\[
	|f(x) - f(y)|\leq L \cdot \norm{x-y} 
.\]
\end{defn}

\begin{note}
	Lipschitz continuity implies continuity and uniform continuity. It is a stronger statement because it tells us \emph{how} the function is (uniformly) continuous. However, it doesn't require differentiability. 
\end{note}

\begin{defn}
For $ 1\leq p < \infty$,
 \[
	 \norm{x}_p = \left( \sum_{ i= 1}^{ n} |x_i|^p \right)^{\frac{1}{p}}  
.\] 
For $ p = \infty$,
\[
\norm{x}_{\infty} = \max_{1\leq i\leq n} |x_i|
.\]
\end{defn}

\begin{remark}
$ \norm{x}_1 $ and $ \norm{x}_2^2 $ have separable terms as they are sums of their components. $ \norm{x}_2^2 $ is also differentiable which makes it the nicest norm to optimize.
\end{remark}

\begin{eg}
	Let $ f: \rr^{n} \to \rr$ be $ L$-Lipschitz continuous w.r.t.  $ \norm{ \cdot }_{\infty} $. Let $ C = [0,1]^{n}$, \emph{i.e.} in $ \rr^{2}$, $ C$ is a square. To solve the problem
	\[
		\min_{x \in C} f(x)
	,\]
	since we have few assumption, there is no better method (in the worst case sense) than the \allbold{uniform grid method}. The idea is that we pick $ p+1$ points in each dimension,  \emph{i.e.} $ \{0,\frac{1}{p},\frac{2}{p},\ldots,1\} $, so we would have $ (p+1)^{n}$ points in total.

	Let $ x^* $ be a global optimal point, then there exists a grid point $ \widetilde{ x}$ s.t.  \[
	\norm{ x^* -\widetilde{ x}}_{\infty} \leq \frac{1}{2} \cdot  \frac{1}{p} 
	.\]
	Thus by Lipschitz continuity, 
	\begin{align*}
		|f(x^* ) - f(\widetilde{ x})| &\leq L \cdot \norm{ x^* -\widetilde{ x}}_{\infty} \\
		&\leq \frac{1}{2} \frac{L}{p}
	\end{align*} 
	So we can find $ \widetilde{ x}$ by taking the discrete minimum of all $ (p+1)^{n}$ grid points.\\

	In (non-discrete) optimization, we usually can't exactly find the minimizer, but rather find something very close.

\begin{defn}
	$ x$ is a  \allbold{$ \epsilon$-optimal solution} to $ \min_{x \in C} f(x)$ if $ x \in C$ and
	\[
		f(x)-f^*  \leq \epsilon
	\]
	where $ f^* = \min_{x \in C} f(x)$.
\end{defn}

Our uniform grid method gives us an $ \epsilon$-optimal solution with $ \epsilon = \frac{L}{2p}$, and requires $ (p+1)^{n}$ function evaluations. Writing $ p$ in terms of  $ \epsilon$, we have $ p=\frac{L}{2 \epsilon}$ so equivalently it requires $ \left( \frac{2L}{ \epsilon} + 1 \right)^{n} $ function evaluations, which approximately is $ \epsilon^{-n}$. 

For $ \epsilon = 10^{-6}$, $ n=100$, it requires  $ 10^{600}$ function evaluations. This is really bad!

Take-aways from this example:
\begin{itemize}
	\item curse-of-dimensionality: there can be trillions of variables in a Google Neural Network. It would be intractable using the grid method.
	\item we need more assumptions to allow us to use more powerful methods.
\end{itemize}
\end{eg}

\subsection{Categorization}
\begin{figure}[H]
	\hspace*{-4cm}
	\includegraphics[width=1.6\textwidth]{./figures/categorization.jpg}
\end{figure}
\newpage
\end{document}
