\documentclass[class=article,crop=false]{standalone} 
%Fall 2020
% Some basic packages
\usepackage{standalone}[subpreambles=true]
\usepackage[utf8]{inputenc}
\usepackage[T1]{fontenc}
\usepackage{textcomp}
\usepackage[english]{babel}
\usepackage{url}
\usepackage{graphicx}
\usepackage{float}
\usepackage{enumitem}
\usepackage{lmodern}
\usepackage{hyperref}
\usepackage[usenames,svgnames,dvipsnames]{xcolor}


\pdfminorversion=7

% Don't indent paragraphs, leave some space between them
\usepackage{parskip}

% Hide page number when page is empty
\usepackage{emptypage}
\usepackage{subcaption}
\usepackage{multicol}
\usepackage[dvipsnames]{xcolor}
\usepackage[b]{esvect}

% Math stuff
\usepackage{amsmath, amsfonts, mathtools, amsthm, amssymb}
\usepackage{bbm}

% Fancy script capitals
\usepackage{mathrsfs}
\usepackage{cancel}
% Bold math
\usepackage{bm}
% Some shortcuts
\newcommand{\rr}{\ensuremath{\mathbb{R}}}
\newcommand{\zz}{\ensuremath{\mathbb{Z}}}
\newcommand{\qq}{\ensuremath{\mathbb{Q}}}
\newcommand{\nn}{\ensuremath{\mathbb{N}}}
\newcommand{\ff}{\ensuremath{\mathbb{F}}}
\newcommand{\cc}{\ensuremath{\mathbb{C}}}
\newcommand{\ee}{\ensuremath{\mathbb{E}}}
\renewcommand\O{\ensuremath{\emptyset}}
\newcommand{\norm}[1]{{\left\lVert{#1}\right\rVert}}
\newcommand{\ve}[1]{{\mathbf{#1}}}
\newcommand\allbold[1]{{\boldmath\textbf{#1}}}
\DeclareMathOperator{\lcm}{lcm}
\DeclareMathOperator{\im}{im}
\DeclareMathOperator{\coim}{coim}
\DeclareMathOperator{\dom}{dom}
\DeclareMathOperator{\tr}{tr}
\DeclareMathOperator{\rank}{rank}
\DeclareMathOperator*{\var}{Var}
\DeclareMathOperator*{\ev}{E}
\DeclareMathOperator{\sinc}{sinc}
\DeclareMathOperator{\dg}{deg}
\DeclareMathOperator{\aff}{aff}
\DeclareMathOperator{\conv}{conv}
\DeclareMathOperator{\epi}{epi}
\DeclareMathOperator{\inte}{int}
\DeclareMathOperator{\ri}{ri}
\DeclareMathOperator*{\argmin}{argmin}
\DeclareMathOperator*{\argmax}{argmax}
\DeclareMathOperator{\graph}{graph}
\DeclareMathOperator{\sgn}{sgn}
\DeclareMathOperator*{\Rep}{Rep}
\DeclareMathOperator{\Proj}{Proj}
\DeclareMathOperator{\prox}{prox}
\DeclareMathOperator{\mat}{mat}
\let\vec\relax
\DeclareMathOperator{\vec}{vec}
\let\Re\relax
\DeclareMathOperator{\Re}{Re}
\let\Im\relax
\DeclareMathOperator{\Im}{Im}
% Put x \to \infty below \lim
\let\svlim\lim\def\lim{\svlim\limits}

%wide hat
\usepackage{scalerel,stackengine}
\stackMath
\newcommand*\wh[1]{%
\savestack{\tmpbox}{\stretchto{%
  \scaleto{%
    \scalerel*[\widthof{\ensuremath{#1}}]{\kern-.6pt\bigwedge\kern-.6pt}%
    {\rule[-\textheight/2]{1ex}{\textheight}}%WIDTH-LIMITED BIG WEDGE
  }{\textheight}% 
}{0.5ex}}%
\stackon[1pt]{#1}{\tmpbox}%
}
\parskip 1ex

%Make implies and impliedby shorter
\let\implies\Rightarrow
\let\impliedby\Leftarrow
\let\iff\Leftrightarrow
\let\epsilon\varepsilon

% Add \contra symbol to denote contradiction
\usepackage{stmaryrd} % for \lightning
\newcommand\contra{\scalebox{1.5}{$\lightning$}}

% \let\phi\varphi

% Command for short corrections
% Usage: 1+1=\correct{3}{2}

\definecolor{correct}{HTML}{009900}
\newcommand\correct[2]{\ensuremath{\:}{\color{red}{#1}}\ensuremath{\to }{\color{correct}{#2}}\ensuremath{\:}}
\newcommand\green[1]{{\color{correct}{#1}}}

% horizontal rule
\newcommand\hr{
    \noindent\rule[0.5ex]{\linewidth}{0.5pt}
}

% hide parts
\newcommand\hide[1]{}

% si unitx
\usepackage{siunitx}
\sisetup{locale = FR}

%allows pmatrix to stretch
\makeatletter
\renewcommand*\env@matrix[1][\arraystretch]{%
  \edef\arraystretch{#1}%
  \hskip -\arraycolsep
  \let\@ifnextchar\new@ifnextchar
  \array{*\c@MaxMatrixCols c}}
\makeatother

\renewcommand{\arraystretch}{0.8}

% Environments
\makeatother
% For box around Definition, Theorem, \ldots
%%fakesection Theorems
\usepackage{thmtools}
\usepackage[framemethod=TikZ]{mdframed}

\theoremstyle{definition}
\mdfdefinestyle{mdbluebox}{%
	roundcorner = 10pt,
	linewidth=1pt,
	skipabove=12pt,
	innerbottommargin=9pt,
	skipbelow=2pt,
	nobreak=true,
	linecolor=blue,
	backgroundcolor=TealBlue!5,
}
\declaretheoremstyle[
	headfont=\sffamily\bfseries\color{MidnightBlue},
	mdframed={style=mdbluebox},
	headpunct={\\[3pt]},
	postheadspace={0pt}
]{thmbluebox}

\mdfdefinestyle{mdredbox}{%
	linewidth=0.5pt,
	skipabove=12pt,
	frametitleaboveskip=5pt,
	frametitlebelowskip=0pt,
	skipbelow=2pt,
	frametitlefont=\bfseries,
	innertopmargin=4pt,
	innerbottommargin=8pt,
	nobreak=false,
	linecolor=RawSienna,
	backgroundcolor=Salmon!5,
}
\declaretheoremstyle[
	headfont=\bfseries\color{RawSienna},
	mdframed={style=mdredbox},
	headpunct={\\[3pt]},
	postheadspace={0pt},
]{thmredbox}

\declaretheorem[%
style=thmbluebox,name=Theorem,numberwithin=section]{thm}
\declaretheorem[style=thmbluebox,name=Lemma,sibling=thm]{lem}
\declaretheorem[style=thmbluebox,name=Proposition,sibling=thm]{prop}
\declaretheorem[style=thmbluebox,name=Corollary,sibling=thm]{coro}
\declaretheorem[style=thmredbox,name=Example,sibling=thm]{eg}

\mdfdefinestyle{mdgreenbox}{%
	roundcorner = 10pt,
	linewidth=1pt,
	skipabove=12pt,
	innerbottommargin=9pt,
	skipbelow=2pt,
	nobreak=true,
	linecolor=ForestGreen,
	backgroundcolor=ForestGreen!5,
}

\declaretheoremstyle[
	headfont=\bfseries\sffamily\color{ForestGreen!70!black},
	bodyfont=\normalfont,
	spaceabove=2pt,
	spacebelow=1pt,
	mdframed={style=mdgreenbox},
	headpunct={ --- },
]{thmgreenbox}

\declaretheorem[style=thmgreenbox,name=Definition,sibling=thm]{defn}

\mdfdefinestyle{mdgreenboxsq}{%
	linewidth=1pt,
	skipabove=12pt,
	innerbottommargin=9pt,
	skipbelow=2pt,
	nobreak=true,
	linecolor=ForestGreen,
	backgroundcolor=ForestGreen!5,
}
\declaretheoremstyle[
	headfont=\bfseries\sffamily\color{ForestGreen!70!black},
	bodyfont=\normalfont,
	spaceabove=2pt,
	spacebelow=1pt,
	mdframed={style=mdgreenboxsq},
	headpunct={},
]{thmgreenboxsq}
\declaretheoremstyle[
	headfont=\bfseries\sffamily\color{ForestGreen!70!black},
	bodyfont=\normalfont,
	spaceabove=2pt,
	spacebelow=1pt,
	mdframed={style=mdgreenboxsq},
	headpunct={},
]{thmgreenboxsq*}

\mdfdefinestyle{mdblackbox}{%
	skipabove=8pt,
	linewidth=3pt,
	rightline=false,
	leftline=true,
	topline=false,
	bottomline=false,
	linecolor=black,
	backgroundcolor=RedViolet!5!gray!5,
}
\declaretheoremstyle[
	headfont=\bfseries,
	bodyfont=\normalfont\small,
	spaceabove=0pt,
	spacebelow=0pt,
	mdframed={style=mdblackbox}
]{thmblackbox}

\theoremstyle{plain}
\declaretheorem[name=Question,sibling=thm,style=thmblackbox]{ques}
\declaretheorem[name=Remark,sibling=thm,style=thmgreenboxsq]{remark}
\declaretheorem[name=Remark,sibling=thm,style=thmgreenboxsq*]{remark*}

\theoremstyle{definition}
\newtheorem{claim}[thm]{Claim}
\theoremstyle{remark}
\newtheorem*{case}{Case}
\newtheorem*{notation}{Notation}
\newtheorem*{note}{Note}
\newtheorem*{motivation}{Motivation}
\newtheorem*{intuition}{Intuition}

% Make section starts with 1 for report type
%\renewcommand\thesection{\arabic{section}}

% End example and intermezzo environments with a small diamond (just like proof
% environments end with a small square)
\usepackage{etoolbox}
\AtEndEnvironment{vb}{\null\hfill$\diamond$}%
\AtEndEnvironment{intermezzo}{\null\hfill$\diamond$}%
% \AtEndEnvironment{opmerking}{\null\hfill$\diamond$}%

% Fix some spacing
% http://tex.stackexchange.com/questions/22119/how-can-i-change-the-spacing-before-theorems-with-amsthm
\makeatletter
\def\thm@space@setup{%
  \thm@preskip=\parskip \thm@postskip=0pt
}

% Fix some stuff
% %http://tex.stackexchange.com/questions/76273/multiple-pdfs-with-page-group-included-in-a-single-page-warning
\pdfsuppresswarningpagegroup=1

\renewcommand{\baselinestretch}{1.5}
\RequirePackage{hyperref}[6.83]
\hypersetup{
  colorlinks=false,
  frenchlinks=false,
  pdfborder={0 0 0},
  naturalnames=false,
  hypertexnames=false,
  breaklinks
}
\urlstyle{same}

\usepackage{graphics}
\usepackage{epstopdf}

%%
%% Add support for color in order to color the hyperlinks.
%% 
\hypersetup{
  colorlinks = true,
  allcolors = siaminlinkcolor,
  urlcolor = siamexlinkcolor,
}
%%fakesection Links
\hypersetup{
    colorlinks,
    linkcolor={red!50!black},
    citecolor={green!50!black},
    urlcolor={blue!80!black}
}
%customization of cleveref
\RequirePackage[capitalize,nameinlink]{cleveref}[0.19]

% Per SIAM Style Manual, "section" should be lowercase
\crefname{section}{section}{sections}
\crefname{subsection}{subsection}{subsections}
\Crefname{section}{Section}{Sections}
\Crefname{subsection}{Subsection}{Subsections}

% Per SIAM Style Manual, "Figure" should be spelled out in references
\Crefname{figure}{Figure}{Figures}

% Per SIAM Style Manual, don't say equation in front on an equation.
\crefformat{equation}{\textup{#2(#1)#3}}
\crefrangeformat{equation}{\textup{#3(#1)#4--#5(#2)#6}}
\crefmultiformat{equation}{\textup{#2(#1)#3}}{ and \textup{#2(#1)#3}}
{, \textup{#2(#1)#3}}{, and \textup{#2(#1)#3}}
\crefrangemultiformat{equation}{\textup{#3(#1)#4--#5(#2)#6}}%
{ and \textup{#3(#1)#4--#5(#2)#6}}{, \textup{#3(#1)#4--#5(#2)#6}}{, and \textup{#3(#1)#4--#5(#2)#6}}

% But spell it out at the beginning of a sentence.
\Crefformat{equation}{#2Equation~\textup{(#1)}#3}
\Crefrangeformat{equation}{Equations~\textup{#3(#1)#4--#5(#2)#6}}
\Crefmultiformat{equation}{Equations~\textup{#2(#1)#3}}{ and \textup{#2(#1)#3}}
{, \textup{#2(#1)#3}}{, and \textup{#2(#1)#3}}
\Crefrangemultiformat{equation}{Equations~\textup{#3(#1)#4--#5(#2)#6}}%
{ and \textup{#3(#1)#4--#5(#2)#6}}{, \textup{#3(#1)#4--#5(#2)#6}}{, and \textup{#3(#1)#4--#5(#2)#6}}

% Make number non-italic in any environment.
\crefdefaultlabelformat{#2\textup{#1}#3}

% My name
\author{Jaden Wang}



\begin{document}
\subsection{Saddle point interpretation [BV 4.2]}
Here we want to find the saddle points as we want to minimize the primal but maximize the dual.
\begin{align*}
	p^* = \min\quad &f_0(x)  \\
	\text{subject to } \quad &f_i(x) \leq 0, i = 1,\ldots,m  \\
&Ax = b  
\end{align*}
This is equivalent to
\begin{align*}
	&\min_{x}f_0(x) + \sup_{\lambda\geq 0, \nu}\left\{ \sum \lambda_i f_i(x) + \nu^{T} (Ax-b)\right\} \\
		 =\ & \min_{x \in D} \sup_{\lambda\geq 0, \nu} \mathscr{L}(x,\lambda,\nu)
\end{align*}
This is because if $ f_i(x)>0$ or  $ a_i^{T} x -b_i \neq 0$ for some $ i$, then we get  $ \infty$ in the supremum, encoding it as infeasible.

Then the dual is
\[
	d^* = \max_{\lambda\geq 0, \nu } g(\lambda,\nu) = \max_{\lambda\geq 0, \nu} \min_{x \in D} \mathscr{L}(x,\lambda,\nu)
.\] 
The weak-duality is equivalent to "min-max" inequality:
\[
	d^* =\max_{\lambda\geq 0,\nu} \min_{x \in D} \mathscr{L}(x,\lambda,\nu) \leq \min_{x \in D} \max_{\lambda\geq 0, \nu} \mathscr{L}(x,\lambda,\nu) = p^* 
.\]
And equality is achieved if strong-duality holds.
\begin{note}
All "min/max" should be "inf/sup" until proven.
\end{note}

\begin{thm}
Saddle point occurs when
\begin{enumerate}[label=(\arabic*)]
	\item strong-duality/strong max/min
	\item inf/sup are achieved.
		
		That is, $ (x^* ,(\lambda^* ,\nu^* ))$ is a saddle point of $ \mathscr{L}(x,(\lambda,\nu))$ if
		\begin{align*}
			\mathscr{L}(x^* ,(\lambda^* ,\nu^* )) &= \inf_x \mathscr{L}(x,(\lambda^* ,\nu^* )) \\
			\mathscr{L}(x^* ,(\lambda^* ,\nu^* )) &= \sup_{\lambda,\nu} \mathscr{L}(x^* ,(\lambda,\nu))
		\end{align*}
\end{enumerate}
\end{thm}

\begin{coro}
	If we know $ \lambda^* ,\nu^* $, then we can find $ x^* $ by solving the unconstrained problem
	\[
		\min_x \mathscr{L}(x,(\lambda^* ,\nu^* ))
	.\] 
\end{coro}
This allows us to solve problems with shared Lagrangians.
\subsubsection{Shared Lagrangian}
\begin{eg}
\begin{align*}
\min\quad &\norm{ x}_1  \\ 
\text{subject to } \quad & \norm{ Ax-b}_2 \leq \epsilon \quad \iff \quad \norm{ Ax-b}_2^2 - \epsilon^2 \leq 0 
\end{align*}
Let
\[
	\mathscr{L}(x,\lambda) = \norm{ x}_1 + \lambda \left( \norm{ Ax-b}_2^2 - \epsilon^2  \right) 
.\]
With the correct $ \lambda^* $, this is equivalent to
\[
\min_x \norm{ x}_1 + \lambda^* \norm{ Ax-b}_2^2
,\]
because dropping the constant doesn't affect minimizer. This unconstrained problem is much nicer because the least squares is differentiable, whereas the original constraint is hard to project.

\end{eg}

Even if we don't know $ \lambda^* $,
\begin{enumerate}[label=(\arabic*)]
	\item guess $ \lambda$, solve $ x = x(\lambda)$, check if the constraint is active, update $ \lambda$ (solve the dual problem).
	\item often $ \epsilon$ is not known (hyper-parameter) and set via cross-validation so we can do cross-validation on $ \lambda$ directly (evaluate trade-off in modeling).
\end{enumerate}
We assume existence of saddle points here, which is given by the following:
\begin{prop}
Slater's on both primal and dual $ \implies$ existence of saddle points. 
\end{prop}

\subsection{Game Theory connection}
Consider a finite, 2-person, 0-sum game: "matrix game" (not Prisoner's dilemma). 

This involves the Minimax Theorem of Von Neumann.
\begin{eg}[rock-paper-scissors]
Player 1 wants to minimize and Player 2 wants to maximize utility. The payoff matrix looks like
\begin{table}[H]
	\centering
	\begin{tabular}{c||c|c|c}
		&P&S&R\\
		\hline
		\hline
		P&0&1&-1\\
		\hline
		S&-1&0&1\\
		\hline
		R&1&-1&0
	\end{tabular}
	\caption*{Row: Player 1; Column: Player 2}
\end{table}

$ u^{T}Pv$ is the payoff, intuitively it means player 1 chooses a row and player 2 chooses a column. For a fair game, the payoff value is 0. Since $ A=-A^{T}$ is antisymmetrical, it's fair. But in reality, $ u$ and  $ v$ actually encode the probability of choose each row/column, which sums up to 1.

Define probability simplex $ \Delta = \{u:u\geq 0, \sum u_i = 1\} $.

\begin{case}[Player 2 knows player 1's strategy]
If $ u$ is known, 
Then the decision is easy: choose $ v \in \argmax_{v \in \Delta} u^{T}Pv$.

If Player 1 knows Player 2 knows Player 1's strategy, then Player 1 should select $ u$ to minimize Player 2's payoff:
 \[
p_1^* = \min_{u \in \Delta} \max_{v \in \Delta} u^{T}Pv
.\]
This is in fact a LP.
\end{case}
\begin{case}[Player 1 knows Player 2's strategy]
\[
p_2^* = \max_{v \in \Delta} \min_{u \in \Delta} u^{T}Pv
.\] 
\end{case}
Intuitively, whoever has knowledge of opponent's move gets an edge, so the payoff when Player 2 has an edge in maximizing will be at least the payoff when Player 1 has an edge in minimizing. That is, $ p_1^* \geq p_2^* $. This is weak duality. Slater's condition for LP requires only a feasible point. Since $ \Delta$ is nonempty, we have strong duality $ p_1^* =p_2^* $.
\end{eg}

\end{document}
