\documentclass[class=article,crop=false]{standalone} 
%Fall 2020
% Some basic packages
\usepackage{standalone}[subpreambles=true]
\usepackage[utf8]{inputenc}
\usepackage[T1]{fontenc}
\usepackage{textcomp}
\usepackage[english]{babel}
\usepackage{url}
\usepackage{graphicx}
\usepackage{float}
\usepackage{enumitem}
\usepackage{lmodern}
\usepackage{hyperref}
\usepackage[usenames,svgnames,dvipsnames]{xcolor}


\pdfminorversion=7

% Don't indent paragraphs, leave some space between them
\usepackage{parskip}

% Hide page number when page is empty
\usepackage{emptypage}
\usepackage{subcaption}
\usepackage{multicol}
\usepackage[dvipsnames]{xcolor}
\usepackage[b]{esvect}

% Math stuff
\usepackage{amsmath, amsfonts, mathtools, amsthm, amssymb}
\usepackage{bbm}

% Fancy script capitals
\usepackage{mathrsfs}
\usepackage{cancel}
% Bold math
\usepackage{bm}
% Some shortcuts
\newcommand{\rr}{\ensuremath{\mathbb{R}}}
\newcommand{\zz}{\ensuremath{\mathbb{Z}}}
\newcommand{\qq}{\ensuremath{\mathbb{Q}}}
\newcommand{\nn}{\ensuremath{\mathbb{N}}}
\newcommand{\ff}{\ensuremath{\mathbb{F}}}
\newcommand{\cc}{\ensuremath{\mathbb{C}}}
\newcommand{\ee}{\ensuremath{\mathbb{E}}}
\renewcommand\O{\ensuremath{\emptyset}}
\newcommand{\norm}[1]{{\left\lVert{#1}\right\rVert}}
\newcommand{\ve}[1]{{\mathbf{#1}}}
\newcommand\allbold[1]{{\boldmath\textbf{#1}}}
\DeclareMathOperator{\lcm}{lcm}
\DeclareMathOperator{\im}{im}
\DeclareMathOperator{\coim}{coim}
\DeclareMathOperator{\dom}{dom}
\DeclareMathOperator{\tr}{tr}
\DeclareMathOperator{\rank}{rank}
\DeclareMathOperator*{\var}{Var}
\DeclareMathOperator*{\ev}{E}
\DeclareMathOperator{\sinc}{sinc}
\DeclareMathOperator{\dg}{deg}
\DeclareMathOperator{\aff}{aff}
\DeclareMathOperator{\conv}{conv}
\DeclareMathOperator{\epi}{epi}
\DeclareMathOperator{\inte}{int}
\DeclareMathOperator{\ri}{ri}
\DeclareMathOperator*{\argmin}{argmin}
\DeclareMathOperator*{\argmax}{argmax}
\DeclareMathOperator{\graph}{graph}
\DeclareMathOperator{\sgn}{sgn}
\DeclareMathOperator*{\Rep}{Rep}
\DeclareMathOperator{\Proj}{Proj}
\DeclareMathOperator{\prox}{prox}
\DeclareMathOperator{\mat}{mat}
\let\vec\relax
\DeclareMathOperator{\vec}{vec}
\let\Re\relax
\DeclareMathOperator{\Re}{Re}
\let\Im\relax
\DeclareMathOperator{\Im}{Im}
% Put x \to \infty below \lim
\let\svlim\lim\def\lim{\svlim\limits}

%wide hat
\usepackage{scalerel,stackengine}
\stackMath
\newcommand*\wh[1]{%
\savestack{\tmpbox}{\stretchto{%
  \scaleto{%
    \scalerel*[\widthof{\ensuremath{#1}}]{\kern-.6pt\bigwedge\kern-.6pt}%
    {\rule[-\textheight/2]{1ex}{\textheight}}%WIDTH-LIMITED BIG WEDGE
  }{\textheight}% 
}{0.5ex}}%
\stackon[1pt]{#1}{\tmpbox}%
}
\parskip 1ex

%Make implies and impliedby shorter
\let\implies\Rightarrow
\let\impliedby\Leftarrow
\let\iff\Leftrightarrow
\let\epsilon\varepsilon

% Add \contra symbol to denote contradiction
\usepackage{stmaryrd} % for \lightning
\newcommand\contra{\scalebox{1.5}{$\lightning$}}

% \let\phi\varphi

% Command for short corrections
% Usage: 1+1=\correct{3}{2}

\definecolor{correct}{HTML}{009900}
\newcommand\correct[2]{\ensuremath{\:}{\color{red}{#1}}\ensuremath{\to }{\color{correct}{#2}}\ensuremath{\:}}
\newcommand\green[1]{{\color{correct}{#1}}}

% horizontal rule
\newcommand\hr{
    \noindent\rule[0.5ex]{\linewidth}{0.5pt}
}

% hide parts
\newcommand\hide[1]{}

% si unitx
\usepackage{siunitx}
\sisetup{locale = FR}

%allows pmatrix to stretch
\makeatletter
\renewcommand*\env@matrix[1][\arraystretch]{%
  \edef\arraystretch{#1}%
  \hskip -\arraycolsep
  \let\@ifnextchar\new@ifnextchar
  \array{*\c@MaxMatrixCols c}}
\makeatother

\renewcommand{\arraystretch}{0.8}

% Environments
\makeatother
% For box around Definition, Theorem, \ldots
%%fakesection Theorems
\usepackage{thmtools}
\usepackage[framemethod=TikZ]{mdframed}

\theoremstyle{definition}
\mdfdefinestyle{mdbluebox}{%
	roundcorner = 10pt,
	linewidth=1pt,
	skipabove=12pt,
	innerbottommargin=9pt,
	skipbelow=2pt,
	nobreak=true,
	linecolor=blue,
	backgroundcolor=TealBlue!5,
}
\declaretheoremstyle[
	headfont=\sffamily\bfseries\color{MidnightBlue},
	mdframed={style=mdbluebox},
	headpunct={\\[3pt]},
	postheadspace={0pt}
]{thmbluebox}

\mdfdefinestyle{mdredbox}{%
	linewidth=0.5pt,
	skipabove=12pt,
	frametitleaboveskip=5pt,
	frametitlebelowskip=0pt,
	skipbelow=2pt,
	frametitlefont=\bfseries,
	innertopmargin=4pt,
	innerbottommargin=8pt,
	nobreak=false,
	linecolor=RawSienna,
	backgroundcolor=Salmon!5,
}
\declaretheoremstyle[
	headfont=\bfseries\color{RawSienna},
	mdframed={style=mdredbox},
	headpunct={\\[3pt]},
	postheadspace={0pt},
]{thmredbox}

\declaretheorem[%
style=thmbluebox,name=Theorem,numberwithin=section]{thm}
\declaretheorem[style=thmbluebox,name=Lemma,sibling=thm]{lem}
\declaretheorem[style=thmbluebox,name=Proposition,sibling=thm]{prop}
\declaretheorem[style=thmbluebox,name=Corollary,sibling=thm]{coro}
\declaretheorem[style=thmredbox,name=Example,sibling=thm]{eg}

\mdfdefinestyle{mdgreenbox}{%
	roundcorner = 10pt,
	linewidth=1pt,
	skipabove=12pt,
	innerbottommargin=9pt,
	skipbelow=2pt,
	nobreak=true,
	linecolor=ForestGreen,
	backgroundcolor=ForestGreen!5,
}

\declaretheoremstyle[
	headfont=\bfseries\sffamily\color{ForestGreen!70!black},
	bodyfont=\normalfont,
	spaceabove=2pt,
	spacebelow=1pt,
	mdframed={style=mdgreenbox},
	headpunct={ --- },
]{thmgreenbox}

\declaretheorem[style=thmgreenbox,name=Definition,sibling=thm]{defn}

\mdfdefinestyle{mdgreenboxsq}{%
	linewidth=1pt,
	skipabove=12pt,
	innerbottommargin=9pt,
	skipbelow=2pt,
	nobreak=true,
	linecolor=ForestGreen,
	backgroundcolor=ForestGreen!5,
}
\declaretheoremstyle[
	headfont=\bfseries\sffamily\color{ForestGreen!70!black},
	bodyfont=\normalfont,
	spaceabove=2pt,
	spacebelow=1pt,
	mdframed={style=mdgreenboxsq},
	headpunct={},
]{thmgreenboxsq}
\declaretheoremstyle[
	headfont=\bfseries\sffamily\color{ForestGreen!70!black},
	bodyfont=\normalfont,
	spaceabove=2pt,
	spacebelow=1pt,
	mdframed={style=mdgreenboxsq},
	headpunct={},
]{thmgreenboxsq*}

\mdfdefinestyle{mdblackbox}{%
	skipabove=8pt,
	linewidth=3pt,
	rightline=false,
	leftline=true,
	topline=false,
	bottomline=false,
	linecolor=black,
	backgroundcolor=RedViolet!5!gray!5,
}
\declaretheoremstyle[
	headfont=\bfseries,
	bodyfont=\normalfont\small,
	spaceabove=0pt,
	spacebelow=0pt,
	mdframed={style=mdblackbox}
]{thmblackbox}

\theoremstyle{plain}
\declaretheorem[name=Question,sibling=thm,style=thmblackbox]{ques}
\declaretheorem[name=Remark,sibling=thm,style=thmgreenboxsq]{remark}
\declaretheorem[name=Remark,sibling=thm,style=thmgreenboxsq*]{remark*}

\theoremstyle{definition}
\newtheorem{claim}[thm]{Claim}
\theoremstyle{remark}
\newtheorem*{case}{Case}
\newtheorem*{notation}{Notation}
\newtheorem*{note}{Note}
\newtheorem*{motivation}{Motivation}
\newtheorem*{intuition}{Intuition}

% Make section starts with 1 for report type
%\renewcommand\thesection{\arabic{section}}

% End example and intermezzo environments with a small diamond (just like proof
% environments end with a small square)
\usepackage{etoolbox}
\AtEndEnvironment{vb}{\null\hfill$\diamond$}%
\AtEndEnvironment{intermezzo}{\null\hfill$\diamond$}%
% \AtEndEnvironment{opmerking}{\null\hfill$\diamond$}%

% Fix some spacing
% http://tex.stackexchange.com/questions/22119/how-can-i-change-the-spacing-before-theorems-with-amsthm
\makeatletter
\def\thm@space@setup{%
  \thm@preskip=\parskip \thm@postskip=0pt
}

% Fix some stuff
% %http://tex.stackexchange.com/questions/76273/multiple-pdfs-with-page-group-included-in-a-single-page-warning
\pdfsuppresswarningpagegroup=1

\renewcommand{\baselinestretch}{1.5}
\RequirePackage{hyperref}[6.83]
\hypersetup{
  colorlinks=false,
  frenchlinks=false,
  pdfborder={0 0 0},
  naturalnames=false,
  hypertexnames=false,
  breaklinks
}
\urlstyle{same}

\usepackage{graphics}
\usepackage{epstopdf}

%%
%% Add support for color in order to color the hyperlinks.
%% 
\hypersetup{
  colorlinks = true,
  allcolors = siaminlinkcolor,
  urlcolor = siamexlinkcolor,
}
%%fakesection Links
\hypersetup{
    colorlinks,
    linkcolor={red!50!black},
    citecolor={green!50!black},
    urlcolor={blue!80!black}
}
%customization of cleveref
\RequirePackage[capitalize,nameinlink]{cleveref}[0.19]

% Per SIAM Style Manual, "section" should be lowercase
\crefname{section}{section}{sections}
\crefname{subsection}{subsection}{subsections}
\Crefname{section}{Section}{Sections}
\Crefname{subsection}{Subsection}{Subsections}

% Per SIAM Style Manual, "Figure" should be spelled out in references
\Crefname{figure}{Figure}{Figures}

% Per SIAM Style Manual, don't say equation in front on an equation.
\crefformat{equation}{\textup{#2(#1)#3}}
\crefrangeformat{equation}{\textup{#3(#1)#4--#5(#2)#6}}
\crefmultiformat{equation}{\textup{#2(#1)#3}}{ and \textup{#2(#1)#3}}
{, \textup{#2(#1)#3}}{, and \textup{#2(#1)#3}}
\crefrangemultiformat{equation}{\textup{#3(#1)#4--#5(#2)#6}}%
{ and \textup{#3(#1)#4--#5(#2)#6}}{, \textup{#3(#1)#4--#5(#2)#6}}{, and \textup{#3(#1)#4--#5(#2)#6}}

% But spell it out at the beginning of a sentence.
\Crefformat{equation}{#2Equation~\textup{(#1)}#3}
\Crefrangeformat{equation}{Equations~\textup{#3(#1)#4--#5(#2)#6}}
\Crefmultiformat{equation}{Equations~\textup{#2(#1)#3}}{ and \textup{#2(#1)#3}}
{, \textup{#2(#1)#3}}{, and \textup{#2(#1)#3}}
\Crefrangemultiformat{equation}{Equations~\textup{#3(#1)#4--#5(#2)#6}}%
{ and \textup{#3(#1)#4--#5(#2)#6}}{, \textup{#3(#1)#4--#5(#2)#6}}{, and \textup{#3(#1)#4--#5(#2)#6}}

% Make number non-italic in any environment.
\crefdefaultlabelformat{#2\textup{#1}#3}

% My name
\author{Jaden Wang}



\begin{document}
\begin{remark}
	IPM are state-of-the-art on problems (used by cvxpy) that are
\begin{enumerate}[label=(\arabic*)]
	\item medium size or smaller (maybe 10000)
	\item conic problems: LP, QP, SOCP, SDP:
		\begin{align*}
		\min\quad & \langle C,X \rangle \\
		\text{subject to } &X \succeq 0 \\
			  & \mathcal{ A}(x) =b
		\end{align*}
		and we can use $ -\log \det(X)$ to satisfy $ X \succeq 0$.
\end{enumerate}
\end{remark}

\subsubsection{(Block) Coordinate Descent, Alternating Minimization, Gauss-Seidel}
This method exploits certain structure of the problem. It's also a "column-action" methods.
\begin{eg}[Gauss-Siedel]
Consider solving the least square problem with $ x \in \rr^{n}$ and let $ G$ be the Gram matrix. The normal equation becomes
\begin{align*}
	Gx &= \widetilde{ b}\\
	\begin{pmatrix} g_1 & \ldots& g_n \end{pmatrix} \begin{pmatrix} x_1\\ \vdots\\ x_n \end{pmatrix} &= \widetilde{ b} \\
	g_i \alpha &= \widetilde{ b} - \left( \sum_{j<i} g_j x_j^{(k+1)}+ \sum_{j>i} g_j x_j^{(k)} \right) \\
	x_i^{(k+1)} &= \alpha \\
\end{align*}
\end{eg}
\begin{remark}
	Jacobi only uses $ x^{(k)}$ for each $ k$, allowing parallelization and randomized order.
\end{remark}

If we do this row-wise, it's called ART (algebraic reconstruction technique) or Kaczmarz algorithm, or POCS (projection onto convex sets).

Consider
\begin{align*}
	\min \quad & f(x), x = \begin{pmatrix} x_1 \\ \vdots\\x_n \end{pmatrix}, x_i  \in C_i \text{ can be blocks} \\
	x_i^{(k+1)} &\in \argmin_{\alpha \in C} f\left( x_1^{(k+1)}, \ldots, x_{j-1}^{(k+1)}, \alpha, x_{i+1}^{(k)},\ldots, x_n^{(k)} \right) \text{ or } \\
	x_i^{(k+1)} &= x_i^{(k)} -\eta \frac{\partial f}{\partial x_i} \left( x_1^{(k+1)},\ldots,x_{i-1}^{(k+1)}, x_i^{(k)},\ldots, x_n^{(k)} \right)  
\end{align*}
The last step is that if it's too hard to find $ \argmin$, we instead just take a gradient at that step.

If we have two variables $ \min f(x,y)$, then
\begin{align*}
	x^{(k+1)} &\in \argmin_{x} f(x,y^{(k)})\\
	y^{(k+1)} & \in \argmin_{y} f(x^{(k+1)},y)
\end{align*}
We can modify it to PALM (proximal alternating linearized minimization) for non-convex problems:
\begin{align*}
	x^{(k+1)} &\in \argmin_{x} f(x,y^{(k)}) + \frac{\mu}{2} \norm{ x - x^{(k)}}^2 \\
	y^{(k+1)} & \in \argmin_{y} f(x^{(k+1)},y) + \frac{\mu}{2} \norm{ y-y^{(k)}}^2 
\end{align*}

\subsubsection{ADMM (Alternating Direction Method of Multipliers)}
See 2011 Boyd et al monograph.

\begin{align*}
\min\quad &f(x) \\
\text{subject to } \quad & Ax=b 
\end{align*}
Attempt 1: Let's try with dual ascent, using $ y$ as the dual variable:
\begin{align*}
	\mathscr{L}(x,y) &= f(x) + \langle y,Ax-b \rangle\\
	g(y) &= \inf_{x} \mathscr{L}(x,y) \\
	x_{k+1} &\in \argmin \mathscr{L}(x,y_k)\\ 
	y_{k+1} &= y_k + t (\underbrace{ Ax_{k+1}-b}_{\nabla g(y_k) } ) 
\end{align*}
This allows us to exploit the separable structure of the original problem if available \emph{e.g.} $ f(x) = \sum f_i(x_i)$, since we need to relax the linear constraint in the original problem, and the dual allows us to make the Lagrangian separable \emph{i.e.} $ \langle y,Ax-b \rangle \implies \langle A^* y,x \rangle - \langle y,b \rangle$. However, the downside is that it may not converge.

Attempt 2: Let's try the augmented Lagrangian which is equivalent to the original problem:
\begin{align*}
	\min \quad &f(x) + \frac{\rho}{2} \norm{ Ax-b}^2 \\
		   &Ax=b
\end{align*}
Unfortunately the Lagrangian is no longer separable due to the quadratic term:
\begin{align*}
	\mathscr{L}(x,y) = f(x) + \langle y,Ax-b \rangle + \frac{\rho}{2} \norm{ Ax-b}^2 
\end{align*}
Would it be possible to combine the two methods?

Attempt 3 (ADMM): let $ F(x) = \sum_{ i= 1}^{ n} f_i(x_i)$ or $ F(v) =  f(x) + g(z)$ if $ n=2$. 
\begin{align*}
	\min \quad & f(x) + g(z)\\
		   & Ax+ Bz = c
\end{align*}
The algorithm is:
\begin{align*}
	x^{(k+1)} &\in \argmin_{x} \mathscr{L}_{\rho} \left( \begin{pmatrix} x\\z^{(k)} \end{pmatrix}, y^{(k)}  \right) \\
	z^{(k+1)} &\in \argmin_{z} \mathscr{L}_{\rho} \left( \begin{pmatrix} x^{(k+1)}\\z \end{pmatrix}, y^{(k)}  \right) \\
	\text{ update } y^{(k+1)} &= y_k + \rho (A x_{k+1}+ B z_{k+1}-c)
\end{align*}
\begin{note}
	If we jointly minimize the first two lines, it becomes the augmented Lagrangian method.
\end{note}

What if $ n>2$, \emph{i.e.} $ \min_x \sum_{ i= 1}^{ n} f_i(x)$, where $ x$ is a block vector of $ x_i$?

One idea is $ \min_{x_i} \sum f_i(x_i) s.t. $ linear constraints enforces $ x_i = x_j$. 
Naive generalization from $ n=2$ doesn't converge very well. Instead we use a consensus trick:
 \begin{align*}
	 F(v) &= G(x) + H(z)
\end{align*}
where $ x= \begin{pmatrix} x_1\\ \vdots \\ x_n \end{pmatrix} $, $ z$ has the same size as  $ x_i$, and $ v = \begin{pmatrix} x\\z \end{pmatrix} $.
\begin{align*}
	\min_{x,z} \quad & F(x) + G(z)\\
			 &\begin{pmatrix} I&&&&-I\\&I&&&-I\\&& \ddots && \vdots \\&&&I&-I \end{pmatrix} \begin{pmatrix} x\\z \end{pmatrix} =0
\end{align*}
This enforces $ x_i = z \implies x_i = x_j$. We see that $ A = I$ and  $ B = \begin{pmatrix} -I\\ \vdots \\ -I \end{pmatrix} $ from the $ n=2$ linear constraint. Now we see $ x_i$ is decoupled at each update step:
\begin{align*}
	x_{k+1} &\in \argmin_{x} \mathscr{L}_{\rho}(x,z,y_k) && \text{ decoupled}\\
	z_{k+1} &\in \argmin_z \mathscr{L}_{\rho} (x_{k+1},z,y_k) = \frac{1}{n} \sum_{ i= 1}^{ n} x_i && \text{ consensus}\\
	\text{ update } &y_{k+1} \text{ as usual} 
\end{align*}
\begin{remark}
	This is a common trick in optimization. In a coupled system, we relax it to be decoupled first and let them recouple later.
\end{remark}
\end{document}
